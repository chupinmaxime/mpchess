
% LTeX: language=en

% mpchess: draw chess boards and positions with MetaPost
%
% Originally written by Maxime Chupin <notezik@gmail.com>,
% 2023.
%
% Distributed under the terms of the GNU free documentation licence:
%   http://www.gnu.org/licenses/fdl.html
% without any invariant section or cover text.

\documentclass[english]{ltxdoc}

\usepackage{tcolorbox}
\tcbuselibrary{listings,breakable}
%\tcbuselibrary{documentation}
\usepackage{enumitem}
\usepackage[tikz]{bclogo}
\usepackage{mflogo}
\usepackage{hologo}
\usepackage{luamplib}
\mplibtextextlabel{enable}
\usepackage{biblatex}
\addbibresource{ctan.bib}
\usepackage{wrapfig}
\usepackage{accsupp}
\usepackage{siunitx}
\usepackage{imakeidx}
%\usepackage{csquotes}
\usepackage{fancyvrb,xparse,xargs}
\usepackage[sfdefault]{FiraSans}
\usepackage[mathrm=sym]{firamath-otf}
\setmonofont{Fira Mono}
%\setmonofont{FiraCode-Regular.ttf}[BoldFont= FiraCode-Bold.ttf,ItalicFont= FiraCode-RegularItalic.otf,BoldItalicFont= FiraCode-BoldItalic.otf,Ligatures={NoCommon, NoDiscretionary, NoHistoric, NoRequired, NoContextual}]

\usepackage{xspace}
\usepackage{animate}
\newcommand{\ctan}{\textsc{ctan}}
\NewDocumentCommand{\package}{ m }{%
  \href{https://ctan.org/pkg/#1}{#1}\xspace
}

\definecolor{darkred}{rgb}{0.6,0.1,0.1}
\definecolor{vert}{rgb}{0.1,0.4,0.1}
\definecolor{bleu}{rgb}{0.2,0.2,0.6}
\definecolor{orange}{rgb}{0.6,0.4,0.}
\colorlet{code}{blue!80!black}

\usepackage[colorlinks=true,urlcolor=orange,linkcolor=orange,menucolor=black,citecolor=orange]{hyperref}

\newcommand \file       {\nolinkurl}
\renewcommand \cmd        {\texttt}
\renewcommand \code   [1] {\texorpdfstring {\texttt{\color{code}#1}} {#1}}
\renewcommand*\cs     [1] {\code{\textbackslash #1}}



\newcommand*\commande{\par\bigskip\noindent\hspace{-30pt}%
  \SaveVerb[aftersave={%
   \UseVerb{Vitem}%
  }%
  ]{Vitem}%
  }
  \newcommand\vitem[1][]{\SaveVerb[%
  aftersave={\item[\textnormal{\UseVerb[#1]{vsave}}]}]{vsave}}
\newcommand*\textme[1]{\textcolor{black}{\rmfamily\textit{#1}}}
%\renewcommand*\meta[1]{% % meta
%  \textme{\ensuremath{\langle}#1\ensuremath{\rangle}}}
\newcommand*\optstar{% % optional star
  \meta{\ensuremath{*}}\xspace}
\DefineShortVerb{\|}
\newcommand\R{\mathbf{R}}
\setlength{\fboxsep}{2pt}
\fvset{%
  codes={\catcode`\«\active \catcode`\×\active },
  defineactive={\makefancyog\makefancytimes},
  formatcom=\color{darkred},
  frame=single
}
% rendre «...» équivalent à \meta{...}
{\catcode`\«\active
  \newcommandx\makefancyog[0][addprefix=\global]{%
    \def«##1»{\meta{##1}}}}
% rendre × équivalent à \optstar
{\catcode`\×\active
  \newcommandx\makefancytimes[0][addprefix=\global]{%
    \def×{\optstar{}}}}


\newcommand\mpchess{\textbf{\textlogo{MP}}\textit{chess}\xspace}



%\addbibresource{biblio.bib}


\lstset{
  numberstyle=\footnotesize\color{vert},
  keywordstyle=\ttfamily\bfseries\color{bleu},
  basicstyle=\ttfamily,
  commentstyle=\itshape\color{vert},
  stringstyle=\ttfamily\color{orange},
  showstringspaces=false,
  language=MetaPost,
  breaklines=true,
  breakindent=30pt,
  defaultdialect=MetaPost,
  classoffset=1,% frame=tb
  morekeywords={init_backboard,set_backboard_width,set_backboard_size,set_color_theme,get_backboard_width,get_backboard_size,set_backboard_width,get_square_dim,set_white_color,set_black_color,set_coords_inside,set_coords_outside,set_coords_font,set_coords,
  set_no_coords,set_white_view,set_black_view,
  set_white_player,set_black_player,set_pieces_theme,set_players_side,init_chessboard,set_empty_chessboard,add_white_pieces,add_black_pieces,clear_areas,clear_files,clear_ranks,clear_squares,build_chessboard_from_fen,build_chessboards_from_pgn,clear_chessboard,
  color_square,draw_arrows,draw_circles,draw_comment,draw_crosses,get_halfmove_number,get_totalmove_number,
  reset_mpchess,set_black_to_move,set_last_move_color,set_comment_color,set_white_to_move,set_whos_to_move,show_last_move,unset_whos_to_move,set_arrow_width,clip_chessboard,build_chessboard_from_fen_file,build_chessboard_from_pgn_file,draw_black_main_lines,draw_black_main_lines_step,draw_white_main_lines,draw_white_main_lines_step,set_main_lines_color,set_possible_moves_color,show_possible_moves,show_possible_moves_step,hide_coordinates,hide_whos_to_move,set_black_squares_color,set_coordinates_font,set_coordinates_position,set_white_squares_color,show_coordinates,show_whos_to_move,set_board_type},
  keywordstyle=\color{darkred},
  classoffset=2,% frame=tb
  morekeywords={backboard,chessboard,chessboard_step},
  keywordstyle=\color{vert},
  classoffset=0,% frame=tb
  morekeywords={draw},
  keywordstyle=\color{bleu}
}

\makeatletter
\tcbset{%
    listing metapost/.code={%
        \def\tcbuselistingtext@input{\begin{mplibcode} input \jobname.listing; \end{mplibcode}}%
    }
}
\makeatother
\newtcblisting[auto counter,]{ExempleMP}[1][]{%
  arc=0pt,outer arc=0pt,
  colback=darkred!3,
  colframe=darkred,
  breakable,fontupper=\small,
  boxsep=0pt,left=5pt,right=5pt,top=5pt,bottom=5pt, bottomtitle =
  3pt, toptitle=3pt,
  boxrule=0pt,bottomrule=0.5pt,toprule=0.5pt, toprule at break =
  0pt, bottomrule at break = 0pt,
  listing side text,
  listing metapost,
  title=Exemple~\thetcbcounter,
  listing options={breaklines},#1
}

\newtcblisting{commandshell}{colback=black,colupper=white,colframe=black,
  arc=0pt,
  listing only,boxsep=0pt,listing
  options={style=tcblatex,language=sh},
  every listing line={\BeginAccSupp{ActualText={}}\textcolor{red}{\small\ttfamily\bfseries user \$> }}\EndAccSupp{}}


  \newtcblisting{mpcode}{
  arc=0pt,outer arc=0pt,
  colback=darkred!3,
  colframe=darkred,
  breakable,
  boxsep=0pt,left=5pt,right=5pt,top=5pt,bottom=5pt, bottomtitle =
  3pt, toptitle=3pt,
  boxrule=0pt,bottomrule=0.5pt,toprule=0.5pt, toprule at break =
  0pt, bottomrule at break = 0pt,
  listing only,boxsep=0pt,listing
  options={breaklines}
}

\newtcblisting{latexcode}{
  arc=0pt,outer arc=0pt,
  colback=darkred!3,
  colframe=darkred,
  breakable,
  boxsep=0pt,left=5pt,right=5pt,top=5pt,bottom=5pt, bottomtitle =
  3pt, toptitle=3pt,
  boxrule=0pt,bottomrule=0.5pt,toprule=0.5pt, toprule at break =
  0pt, bottomrule at break = 0pt,
  listing only,boxsep=0pt,listing
  options={breaklines,language={[LaTeX]TeX}}
}


\newcommand\pdf{\textsc{pdf}}


\newfontfamily{\chessfont}{mpchess font}
\newcommand\bP{{\chessfont \char"265F}} % black Pawn
\newcommand\bN{{\chessfont \char"265E}} % black Knight
\newcommand\bB{{\chessfont \char"265D}} % black Bishop
\newcommand\bR{{\chessfont \char"265C}} % black Rook
\newcommand\bQ{{\chessfont \char"265B}} % black Queen
\newcommand\bK{{\chessfont \char"265A}} % black King
\newcommand\wP{{\chessfont \char"2659}} % white Pawn
\newcommand\wN{{\chessfont \char"2658}} % white Knight
\newcommand\wB{{\chessfont \char"2657}} % white Bishop
\newcommand\wR{{\chessfont \char"2656}} % white Rook
\newcommand\wQ{{\chessfont \char"2655}} % white Queen
\newcommand\wK{{\chessfont \char"2654}} % white King

\usepackage[english]{babel}

\makeindex[title=Command Index, columns=2]



%\lstset{moredelim=*[s][\color{red}\rmfamily\itshape]{<}{>}}
%\lstset{moredelim=*[s][\color{blue}\rmfamily\itshape]{<<}{>>}}

\begin{document}

\title{{MPchess} : drawing chess boards and positions with \hologo{METAPOST}}
\author{Maxime Chupin, \url{notezik@gmail.com}}
\date{\today}

%% === Page de garde ===================================================
\thispagestyle{empty}
\begin{tikzpicture}[remember picture, overlay]
  \node[below right, shift={(-4pt,4pt)}] at (current page.north west) {%
    \includegraphics{fond.pdf}%
  };
\end{tikzpicture}%

\noindent
{\Huge \mpchess}\par\medskip
\noindent
{\Large  drawing chess boards and positions with \hologo{METAPOST}}\\[1cm]
\parbox{0.6\textwidth}{
  \begin{mplibcode}
    input mpchess
    string pgnstr;
    pgnstr:="1. e4 e5 2. Bc4 d6 3. Nf3 Bg4 4. Nc3 g6 5. Nxe5 Bxd1 ";
    build_chessboards_from_pgn(pgnstr);
    beginfig(0);
    set_backboard_width(8cm);
    init_backboard;
    draw backboard;
    show_last_move(10);
    draw_comment("?","d1");
    color_square(0.3[green,black])("c4","c3","e5");
    color_square(0.3[red,black])("e8");
    draw chessboard_step(10);
    draw_arrows(0.3[green,black])("e5|-f7","c3-|d5");
    draw_arrows(0.3[red,black])("c4--f7");
    endfig;
  \end{mplibcode}
}\hfill
\parbox{0.5\textwidth}{\Large\raggedleft
  \textbf{Contributor}\\
  Maxime \textsc{Chupin}\\
  \url{notezik@gmail.com}
}
\vfill
\begin{center}
  Version 0.7, 2023, July, 20th \\
  \url{https://plmlab.math.cnrs.fr/mchupin/mpchess}
\end{center}
%% == Page de garde ====================================================
\newpage

%\maketitle

\begin{abstract}
  The \mpchess package allows you to draw chess boards and positions.
  The appearance of its drawings is modern and largely inspired by what is offered by
  the excellent web site \url{Lichess.org}.
  Relying on \MP{} probably allows more graphic flexibility than the
  excellent \LaTeX{} chess packages that already exist.
\end{abstract}


\begin{center}
  \url{https://plmlab.math.cnrs.fr/mchupin/mpchess}
  \url{https://github.com/chupinmaxime/mpchess}
\end{center}

\tableofcontents

\bigskip

\begin{tcolorbox}[ arc=0pt,outer arc=0pt,
  colback=darkred!3,
  colframe=darkred,
  breakable,
  boxsep=0pt,left=5pt,right=5pt,top=5pt,bottom=5pt, bottomtitle =
  3pt, toptitle=3pt,
  boxrule=0pt,bottomrule=0.5pt,toprule=0.5pt, toprule at break =
  0pt, bottomrule at break = 0pt,]
  \itshape
  This package is in beta version---do not hesitate to report bugs, as well as requests for improvement.
\end{tcolorbox}

\section{Installation}

\mpchess is on \ctan{} and can also be installed via the package manager of your
distribution.

\begin{center}
  \url{https://www.ctan.org/pkg/mpchess}
\end{center}


\subsection{With \TeX live under Linux or macOS}

To install \mpchess with \TeX Live, you will have to create the directory
\lstinline+texmf+ directory in your \lstinline+home+. 

\begin{commandshell}
mkdir ~/texmf
\end{commandshell}

Then, you will have to place the \lstinline+.mp+ files in the 
\begin{center}
  \lstinline+~/texmf/metapost/mpchess/+
\end{center}

\mpchess consists of 7 files \hologo{METAPOST} :
\begin{itemize}
  \item \verb+mpchess.mp+;
  \item \verb+mpchess-chessboard.mp+;
  \item \verb+mpchess-pgn.mp+;
  \item \verb+mpchess-fen.mp+;
  \item \verb+mpchess-cburnett.mp+;
  \item \verb+mpchess-pieces.mp+;
  \item \verb+mpchess-skak.mp+.
\end{itemize}

Once this is done, \mpchess will be loaded with the classic \MP
input code
\begin{mpcode}
input mpchess
\end{mpcode}

\subsection{With Mik\TeX{} and Windows}

These two systems are unknown to the author of \mpchess, so we
refer you to the Mik\TeX documentation concerning the addition of local packages:
\begin{center}
  \url{http://docs.miktex.org/manual/localadditions.html}
\end{center}



\subsection{Dependencies}


\mpchess depends, of course on \MP, as well as the packages \package{hatching}
and---if \mpchess is not used with \hologo{LuaLaTeX} and the \package{luamplib}
package---the \package{latexmp} package.


\section{Package Rationale and General Philosophy}

There are already \LaTeX{} packages for drawing chess boards and positions,
including the very good package \package{xskak}~\cite{ctan-xskak}coupled with
the package \package{chessboard}~\cite{ctan-chessboard}. 
There, Ulrike Fisher ade improvements, undertaken maintenance work, and provided us with excellent tools to make chess diagrams and to handle the different
formats of
game descriptions\footnote{She even developed the
package to handle various chess fonts.}. The
documentation of each of these packages is very good.

Several things motivated the creation of \mpchess. First of all, with
\package{chessboard} the addition of a set of pieces is not very easy, because
it relies on fonts. Moreover, the author finds that drawing chess game diagrams is by its nature a very graphical task, and that using a dedicated drawing
language offers increased flexibility. In that case, what better than \MP~\cite{ctan-metapost}?

With \mpchess, the final image of the chess board is built with the pieces
by successive layers. Thus, we begin by producing and drawing the board
(\lstinline+backboard+), which we can modify---for example, by
coloring some squares. We then can add the pieces of the position (\lstinline+chessboard+),
and finally, we can annotate the whole thing with marks, colors,
arrows, and so forth. 


Moreover, \mpchess produces images that are graphically close to what is
provided by excellent  \emph{open source} website \url{https://lichess.org}. The colors, the pieces and the general aspect of
\mpchess are largely inspired
by what this website offers.

\section{Board}

The board is called with \mpchess{} \lstinline+backboard+.
You have to initialize the board before drawing it. This is done with the
following command:\par

\commande|init_backboard|\smallskip\index{init_backboard@\lstinline+init_backboard+}

This command constructs a \MP{} \lstinline+picture+ named
\mbox{\lstinline+backboard+.} It should then be drawn as shown in the following example.
\begin{ExempleMP}
input mpchess

beginfig(0);
init_backboard;
draw backboard;
endfig;
\end{ExempleMP}

This initialization will make it possible to use the
various options and features that are described in the
following sections.

\subsection{Size of the Board}

When creating the \lstinline+backboard+, you can further specify its width. This is done with the following command:\par\bigskip

\commande|set_backboard_width(«dim»)|\smallskip\index{set_backboard_width@\lstinline+set_backboard_width+}

\begin{description}
  \item[\meta{dim}:] is the width of the board (with units). By default, this dimension is \SI{5}{cm}.
\end{description}

The use of this command is illustrated in the example~\ref{ex:widthcase}. This
command should be used before \lstinline+init_backboard+ so that it is taken
into account when creating the \lstinline+picture+. 
\bigskip

The size of the board can be retrieved with the following command:

\commande|get_backboard_width|\smallskip\index{get_backboard_width@\lstinline+get_backboard_width+}

This command returns a \lstinline+numeric+.

\subsection{Number of Squares}

By default, the game board contains 64 squares ($8\times 8$). You can change
this with the following command: 
\commande|set_backboard_size(«nbr»)|
\smallskip\index{set_backboard_size@\lstinline+set_backboard_size+}

\begin{description}
  \item[\meta{nbr}:] is the desired number of squares. The board will then be
  square of size \meta{nbr}$\times$\meta{nbr}. By default this number is 8. 
\end{description}

Again, this command must be used before \lstinline+init_backboard+ for it to be
taken into account, as shown in the following example. 

\begin{ExempleMP}[label=ex:widthcase]
input mpchess

beginfig(0);
set_backboard_width(3cm);
set_backboard_size(6);
init_backboard;
draw backboard;
endfig;
\end{ExempleMP}

To specify size of the game board, you can use the following command:
 

\commande|get_backboard_size|\smallskip\index{get_backboard_size@\lstinline+get_backboard_size+}

This command returns a \lstinline+numeric+.

\subsection{Dimension of a Square}

Depending on the number of squares on the board and the prescribed width of the board, \mpchess calculates the dimension (width or height) of a square. This
serves as a general unit. To obtain it, use the following command.

\commande|get_square_dim|\smallskip\index{get_square_dim@\lstinline+get_square_dim+}

This command returns a \lstinline+numeric+.

\subsection{Setting the Color Theme}

\subsubsection{Predefined Color Themes}

Several color themes are available. To choose a color theme, use the following
command:

\commande|set_color_theme(«string»)|\smallskip\index{set_color_theme@\lstinline+set_color_theme+}

\begin{description}
  \item[\meta{string}] can be:
\begin{itemize}
\item \lstinline+"BlueLichess"+ (default);
\item \lstinline+"BrownLichess"+;
\item \lstinline+"GreenLichess"+;
\item \lstinline+"PinkPyramidalLichess"+;
\item \lstinline+"Wood"+;
\item or \lstinline+"Classical"+.
\end{itemize}
\end{description}

The following example shows the results obtained from \lstinline+"BrownLichess"+:
\begin{ExempleMP}
input mpchess
beginfig(0);
set_color_theme("BrownLichess");
init_backboard;
draw backboard;
endfig;
\end{ExempleMP}

The table~\ref{tab:color} shows the different results of the different themes. 

\begin{table}
  \centering
\begin{tabular}{cc}

\lstinline+"GreenLichess"+ theme&\lstinline+"Classical"+ theme\\
\begin{mplibcode}
  input mpchess
  beginfig(0);
  init_chessboard;
  set_backboard_width(4cm);
  set_color_theme("GreenLichess");
  init_backboard;
  draw backboard;
  endfig;
\end{mplibcode}
&\begin{mplibcode}
  input mpchess
  beginfig(0);
  init_chessboard;
  set_backboard_width(4cm);
  set_color_theme("Classical");
  init_backboard;
  draw backboard;
  endfig;
\end{mplibcode}\\
\lstinline+"PinkPyramidalLichess"+ theme& \lstinline+"Wood"+ theme\\
\begin{mplibcode}
  input mpchess
  beginfig(0);
  init_chessboard;
  set_backboard_width(4cm);
  set_color_theme("PinkPyramidalLichess");
  init_backboard;
  draw backboard;
  endfig;
\end{mplibcode}&
\begin{mplibcode}
  input mpchess
  beginfig(0);
  init_chessboard;
  set_backboard_width(4cm);
  set_color_theme("Wood");
  init_backboard;
  draw backboard;
  endfig;
\end{mplibcode}\\
\end{tabular}
\caption{The different color themes provided by \mpchess.}\label{tab:color}
\end{table}



\subsubsection{Configuring a Personal Color Theme}

\paragraph{Colors.}
A color theme is  the definition of two colors.
These can be defined with the following commands\footnote{Attention,
in version 0.6, \lstinline+set_white_color+ became
\lstinline+set_white_squares_color+ and \lstinline+set_black_color+ became
\lstinline+set_black_squares_color+.}:

\commande|set_white_squares_color(«color»)|\index{set_white_squares_color@\lstinline+set_white_squares_color+}\par
\commande|set_black_squares_color(«color»)|\index{set_black_squares_color@\lstinline+set_black_squares_color+}\smallskip

\meta{color} is a \MP{} \lstinline+color+.

\begin{ExempleMP}
input mpchess
beginfig(0);
set_white_squares_color((0.9,0.8,0.8));
set_black_squares_color((0.7,0.6,0.6));
init_backboard;
draw backboard;
endfig;
\end{ExempleMP}

\paragraph{Color Types.}

To choose a type, you can use the following command:

\commande|set_board_type(«string»)|\index{set_board_type@\lstinline+set_board_type+}\smallskip


Three types of coloring are available:

\begin{description}
  \item[\meta{string}] can be:
\begin{itemize}
\item \lstinline+"flat"+,  simple flat coloring (default);
\item \lstinline+"pyramidal"+, Lichess \emph{pyramidal} coloring; 
\item \lstinline+"wood"+, wood imitation.
\end{itemize}
\end{description}

Here is an example coupling color and type definitions. 

\begin{ExempleMP}
input mpchess
beginfig(0);
set_white_squares_color((0.9,0.8,0.8));
set_black_squares_color((0.7,0.6,0.6));
set_board_type("wood");
init_backboard;
draw backboard;
endfig;
\end{ExempleMP}
  


\subsection{Display Coordinates}

You may have noticed in the various examples that by default, the coordinates are, as on the Lichess site, written in small letters inside the boxes.

\mpchess allows you to choose the coordinates position inside  or outside the board with the
following command\footnote{Attention,
in version 0.6, \lstinline+set_coords_inside+ and \lstinline+set_coords_outside+ became
\lstinline+set_coordinates_position+.}:

\commande|set_coordinates_position(«string»)|\index{set_coordinates_position@\lstinline+set_coordinates_position+}\smallskip


\begin{description}
  \item[\meta{string}] can be:
\begin{itemize}
\item \lstinline+"inside"+ (default);
\item \lstinline+"outside"+.
\end{itemize}
\end{description}

The result is as follows.



\begin{ExempleMP}
input mpchess
beginfig(0);
set_coordinates_position("outside");
init_backboard;
draw backboard;
endfig;
\end{ExempleMP}


You can see in the previous examples that with \package{luamplib} and
\LaTeX, the font used for the coordinates is the font of the current document. To draw these letters and
these numbers, \mpchess uses the \MP{} operator  \lstinline+infont+ and the font
is set to \lstinline+defaultfont+ by default\footnote{With \package{luamplib}
the \lstinline+infont+ operator is redefined and its argument is simply
ignored}, so it is not possible to modify the composition font of the
coordinates. This font can be changed with the following command\footnote{Attention,
in version 0.6, \lstinline+set_coords_font+ became 
\lstinline+set_coordinates_font+.}.

\commande|set_coordinates_font(«font»)|\index{set_coordinates_font@\lstinline+set_coordinates_font+}\smallskip

It will then be necessary to use the naming conventions specific to the \MP{}
operator
\lstinline+infont+, and we refer to the
\MP documentation~\cite{ctan-metapost} for more details.

You can also delete the coordinates with the following command\footnote{Attention,
in version 0.6, \lstinline+set_no_coords+ became
\lstinline+hide_coordinates+ and \lstinline+set_coords+ became
\lstinline+show_coordinates+.}:

\commande|hide_coordinates|\index{hide_coordinates@\lstinline+hide_coordinates+}\smallskip

And the reverse command also exists:

\commande|show_coordinates|\index{show_coordinates@\lstinline+show_coordinates+}\smallskip


\subsection{White or Black Side}

To choose if you want to see the tray on the white or black side, \mpchess
provides two commands:

\commande|set_white_view|\index{set_white_view@\lstinline+set_white_view+}\smallskip

\commande|set_black_view|\index{set_black_view@\lstinline+set_black_view+}\smallskip

(By default, we see the board on the white side.)

\subsection{Players' Names}

You can fill in the names of the players so that they are noted around the
chessboard. This is done with the following commands: 

\commande|set_white_player(«string»)|\index{set_white_player@\lstinline+set_white_player+}\smallskip

\commande|set_black_player(«string»)|\index{set_black_player@\lstinline+set_black_player+}\smallskip

\begin{description}
  \item[\meta{string} :] is the string interpreted by \LaTeX{} to display.
\end{description}

\begin{ExempleMP}
input mpchess
beginfig(0);
set_white_player("\textbf{GM} Kasparov");
set_black_player("\textbf{GM} Kramnik");

init_backboard;
draw backboard;
endfig;
\end{ExempleMP}

It is possible to place the names on the right side of the board without the
black bands present by default. 
This happens either if the coordinates are printed outside the tray, or if the
following command is used:

\commande|set_players_side|\index{set_players_side@\lstinline+set_players_side+}\smallskip

\begin{ExempleMP}[righthand width=0.6\linewidth]
input mpchess
beginfig(0);
set_white_player("\textbf{GM} Kasparov");
set_black_player("\textbf{GM} Kramnik");
set_players_side;
init_backboard;
draw backboard;
endfig;
\end{ExempleMP}



\section{Pieces and Positions}

\mpchess, as described above, builds the picture of a chess position
layer by layer. This section describes the configuration of pieces and
positions.

Internally, \mpchess builds a table on the board grid. Then, some
macros allow to generate a \lstinline+picture+ to be drawn \emph{over} the board (\lstinline+backboard+).

\subsection{Setting the Theme of the Pieces}

\mpchess provides for the moment three themes of pieces. The default theme is
called \lstinline+mpchess+. It has been designed for this package \hologo{METAPOST}.
It has been proposed to the Lichess project, and has been accepted. Thus, you
will also have access to the \lstinline+mpchess+ piece set with
Lichess\footnote{The open-source projects feed each other! Even if obviously,
this package has borrowed much more from Lichess than the other way around.}

Another theme is borrowed from
Lichess (\lstinline+cburnett+) and the last one is borrowed from the
package~\package{skak}~cite{ctan-skak}\footnote{Which provides the \MF{} code
for a chess font, which has been easily adapted to \MP{} for \mpchess.}.

To choose the theme, use the following command:

\commande|set_pieces_theme(«string»)|\index{set_pieces_theme@\lstinline+set_pieces_theme+}\smallskip

\begin{description}
  \item[\meta{string}:] can be:
\begin{itemize}
\item \lstinline+"mpchess"+  (default), to use the set specially designed for this
package;
\item \lstinline+"cburnett"+, to use the Lichess \emph{cburnett}
pieces set;
\item \lstinline+"skak"+, to use the \package{skak}
pieces set.
\end{itemize}
\end{description}


The table~\ref{tab:pieces} shows the result of the three sets of pieces.
\begin{table}
  \centering
\begin{tabular}{cc}

\lstinline+cburnett+ theme&\lstinline+mpchess+ theme\\
\begin{mplibcode}
  input mpchess
  beginfig(0);
  init_chessboard;
  set_backboard_width(4cm);
  set_pieces_theme("cburnett");
  hide_whos_to_move;
  init_backboard;
  draw backboard;
  draw chessboard_step(0);
  endfig;
\end{mplibcode}
&\begin{mplibcode}
  input mpchess
  beginfig(0);
  init_chessboard;
  set_backboard_width(4cm);
  set_pieces_theme("mpchess");
  hide_whos_to_move;
  init_backboard;
  draw backboard;
  draw chessboard_step(0);
  endfig;
\end{mplibcode}\\
\lstinline+skak+ theme&\\
\begin{mplibcode}
  input mpchess
  beginfig(0);
  init_chessboard;
  set_backboard_width(4cm);
  set_pieces_theme("skak");
  hide_whos_to_move;
  init_backboard;
  draw backboard;
  draw chessboard_step(0);
  endfig;
\end{mplibcode}&\\
\end{tabular}
\caption{The different themes of pieces provided by \mpchess.}\label{tab:pieces}
\end{table}


\subsection{Specifying to Move}

\mpchess indicates which player has the current move. This is done by
a small colored triangle (white or black) at the end of the board (which
you can see in the following examples).

To specify which side is to move, use the following commands:

\commande|set_white_to_move|\index{set_white_to_move@\lstinline+set_white_to_move+}\smallskip


\commande|set_black_to_move|\index{set_black_to_move@\lstinline+set_black_to_move+}\smallskip

By default, white is to move, and this information is displayed.

To activate or deactivate this display, use one of the following two
commands\footnote{Attention,
in version 0.6, \lstinline+set_whos_to_move+ and
\lstinline+unset_whos_to_move+ became \lstinline+show_whos_to_move+ and
\lstinline+hide_whos_to_move+.}:

\commande|show_whos_to_move|\index{show_whos_to_move@\lstinline+show_whos_to_move+}\smallskip


\commande|hide_whos_to_move|\index{hide_whos_to_move@\lstinline+hide_whos_to_move+}\smallskip

\subsection{Draw a Position}

The commands described below allow you to build a position in several ways
(adding pieces one by one, reading a \textsc{fen} file, etc.). Once a position
has been constructed, it can be plotted 
using the following command, which generates a \MP{} \lstinline+picture+.


\commande|chessboard|\index{chessboard@\lstinline+chessboard+}\smallskip \label{com:chessboard}

(The use of this command will be widely illustrated in the following examples.)

\subsection{Build a position}

\subsubsection{Initialization}

To obtain the initial position of a game, simply use the following command:

\commande|init_chessboard|\index{init_chessboard@\lstinline+init_chessboard+}\smallskip

\begin{ExempleMP}
input mpchess
beginfig(0);
init_backboard;
draw backboard;
init_chessboard;
draw chessboard;
endfig;
\end{ExempleMP}

You can also initialize an empty \lstinline+chessboard+ with the following command:

\commande|set_empty_chessboard|\index{set_empty_chessboard@\lstinline+set_empty_chessboard+}\smallskip

\subsubsection{Adding Pieces}

You can add pieces to build a position with the following two commands:

\commande|add_white_pieces(«piece1»,«piece2»,etc.)|\index{add_white_pieces@\lstinline+add_white_pieces+}\smallskip


\commande|add_black_pieces(«piece1»,«piece2»,etc.)|\index{add_black_pieces@\lstinline+add_black_pieces+}\smallskip

These commands take lists of \textbf{\meta{piece}}, which are strings
that describe the piece and the position using
algebraic notation. There is no limitation on the number of pieces in the list.

The following example illustrates the use of these commands:

\begin{ExempleMP}
input mpchess
beginfig(0);
init_backboard;
draw backboard;
set_empty_chessboard;
add_white_pieces("e1","Kd2");
add_black_pieces("e7","f6","Kb8");
draw chessboard;
endfig;
\end{ExempleMP}


\subsubsection{Deleting Pieces}

\mpchess provides several commands to remove items from a position.

The first command allows you to delete an item from a square. This command takes
a list of squares, using algebraic notation:

\commande|clear_squares(«square1»,«square2»,etc.)|\index{clear_squares@\lstinline+clear_squares+}\smallskip

The variables \textbf{\meta{square1}}, \textbf{\meta{square2}},
and so forth are strings; for example, \lstinline+"a3"+.
\medskip

The following command deletes a set of squares in a region
determined by two coordinates on the board. This command may take
a list of regions:

\commande|clear_areas(«area1»,«area2»,etc.)|\index{clear_areas@\lstinline+clear_areas+}\smallskip

The variables \textbf{\meta{area1}}, \textbf{\meta{area2}}, and so
forth are strings consisting of two coordinates separated by a hyphen; for example, \lstinline+"a3-g7"+.
\medskip

The following command deletes all the cells in a file (column) determined by a
letter on the board. This command may take a list of files:


\commande|clear_files(«file1»,«file2»,etc.)|\index{clear_files@\lstinline+clear_files+}\smallskip

The variables \textbf{\meta{file1}}, \textbf{\meta{file2}}, and so
forth are strings of characters consisting of a letter; for example, \lstinline+"a"+.
\medskip

The following command deletes all the cells in a rank (line) determined by a
number on the board. This command may take a list of ranks.

\commande|clear_ranks(«rank1»,«rank2»,etc.)|\index{clear_ranks@\lstinline+clear_ranks+}\smallskip

The variables \textbf{\meta{rank1}}, \textbf{\meta{rank2}}, and so
forth are strings made up of a number; for example, \lstinline+"4"+.

The use of all these commands is illustrated in the following example:

\begin{ExempleMP}
input mpchess
beginfig(0);
init_backboard;
draw backboard;
init_chessboard;
clear_squares("a1","b2");
clear_areas("c2-d7");
clear_files("f","h");
clear_ranks("8");
draw chessboard;
endfig;
\end{ExempleMP}



\subsection{Reading Data in \textsc{fen} Format}

\mpchess allows you to read a position in the \textsc{fen} format thanks to the following command:

\commande|build_chessboard_from_fen(«string»)|\index{build_chessboard_from_fen@\lstinline+build_chessboard_from_fen+}\smallskip

\begin{description}
\item[\meta{string}:] is a string describing a position in
\textsc{fen} format. Note that all information after the 
\texttt{w} or \texttt{b} character is ignored. 
\end{description}

\begin{ExempleMP}
input mpchess;
beginfig(0);
init_backboard;
draw backboard;
string fenstr;
fenstr := "7r/2p1kp1p/p1B2p2/1pb5/8/2PP4/PP1N1PPP/R5K1 b - - 2 19";
build_chessboard_from_fen(fenstr);
draw chessboard;
endfig;
\end{ExempleMP}


It is also possible to read an external file containing on its first line a
string in \textsc{fen} format with the following command:

\commande|build_chessboard_from_fen_file(«string»)|\index{build_chessboard_from_fen_file@\lstinline+build_chessboard_from_fen_file+}\smallskip

\begin{description}
  \item[\meta{string}:] is a string of characters (between double quotes)
  indicating the name of the file to be read.  
\end{description}


\begin{ExempleMP}
input mpchess;
beginfig(0);
init_backboard;
draw backboard;
build_chessboard_from_fen_file("test.fen");
draw chessboard;
endfig;
\end{ExempleMP}

\subsection{Reading Data in \textsc{pgn} Format}

\mpchess also reads strings in the \textsc{pgn} format.
Please note, this is a partial implementation of the format----in particular, \mpchess
does not manage the \emph{tags} of the \textsc{pgn} format. Rather, \mpchess handles only
the string describing the moves played. In the same way, the accepted format
 by \mpchess does not allow variants or comments.

When such a functionality is used, \mpchess stores all the
intermediate positions and thus represents them.

To construct the positions, use the following command:

\commande|build_chessboards_from_pgn(«string»)|\index{build_chessboards_from_pgn@\lstinline+build_chessboards_from_pgn+}\smallskip


Once the positions are built, they can be represented with the following command:

\commande|chessboard_step(«int»)|\index{chessboard_step@\lstinline+chessboard_step+}\smallskip
\begin{description}
\item[\meta{int}:] is the number of the step. The initial configuration is numbered 0, and then each move, white or black, is numbered.
\end{description}

This command, like \lstinline+chessboard+ (see page~\pageref{com:chessboard}),
returns a \lstinline+picture+. 

The following example illustrates the use of these commands:

\begin{ExempleMP}
input mpchess;
string pgnstr;
pgnstr := "1. e4 e5 2. Nf3 Nc6 3. Nxe5 Nxe5 4. Bb5 c6";
build_chessboards_from_pgn(pgnstr);
beginfig(0);
init_backboard;
draw backboard;
draw chessboard_step(3); % Nf3
endfig;
\end{ExempleMP}


It is also possible to read an external file containing on its first line a
string in  \textsc{pgn} format with the following command:

\commande|build_chessboard_from_pgn_file(«string»)|\index{build_chessboard_from_pgn_file@\lstinline+build_chessboard_from_pgn_file+}\smallskip

\begin{description}
  \item[\meta{string}:] is a string of characters (between double quotes),
  indicating the name of the file to read.  
\end{description}

\begin{ExempleMP}
input mpchess;
build_chessboards_from_pgn_file("test.pgn");
beginfig(0);
init_backboard;
draw backboard;
draw chessboard_step(4); % Nc6
endfig;
\end{ExempleMP}

\subsubsection{Showing the Last Move}

The last move can be displayed automatically with the following command:

\commande|show_last_move(«int»)|\index{show_last_move@\lstinline+show_last_move+}\smallskip

\begin{description}
\item[\meta{int}:] is the number of the step. The initial setup is numbered 0, and then each move, white or black, is numbered.
\end{description}

This command uses transparent colors to show the two squares where the last move
started and ended. Thus, it must be used between the drawing of the board
(\lstinline+draw backboard+) and the drawing of the pieces 
(\lstinline+draw chessboard_step(i)+).

\begin{ExempleMP}
input mpchess;
string pgnstr;
pgnstr := "1. e4 e5 2. Nf3 Nc6 3. Nxe5 Nxe5 4. Bb5 c6";
build_chessboards_from_pgn(pgnstr);
beginfig(0);
init_backboard;
draw backboard;
show_last_move(3);
draw chessboard_step(3); % Nf3
endfig;
\end{ExempleMP}

You can configure the color used to color the squares of the last
move with the following command:

\commande|set_last_move_color(«color»)|\index{set_last_move_color@\lstinline+set_last_move_color+}\smallskip
\begin{description}
  \item[\meta{color}:] is a \MP{} \lstinline+color+.
\end{description}



\subsubsection{Getting the Number of Moves}

You can get the number of half moves with the following command:

\commande|get_halfmove_number|\index{get_halfmove_number@\lstinline+get_halfmove_number+}\smallskip

This command returns a \lstinline+numeric+.

You can also get the total number of moves---in the sense that they are numbered
in the \textsc{pgn} format---with the following command:

\commande|get_totalmove_number|\index{get_totalmove_number@\lstinline+get_totalmove_number+}\smallskip

This command returns a \lstinline+numeric+.

\section{Annotation}

Numerous commands allow you to annotate the chessboard (arrow, color, circle, cross,
etc.).

\subsection{Arrows}

The command for drawing arrows on the chessboard is the following:

\commande|draw_arrows(«color»)(«string1»,«string2», etc.)|\index{draw_arrows@\lstinline+draw_arrows+}\smallskip
\begin{description}
\item[\meta{color}:] is a \MP{} \lstinline+color+.
\item[\meta{string1}:] is a string (between double-quotes) consisting of two
coordinates (letter and number) separated by two characters that can be 
\begin{description}
\vitem+--+ to connect the two squares in a straight line;
\vitem+-|+ to connect the two squares in a broken line, first horizontally then
vertically; 
\vitem+|-+ to connect the two squares in a broken line, first vertically then
horizontally. 
\end{description}
\end{description}

The following example illustrates the use of this command:

\begin{ExempleMP}
input mpchess;
string pgnstr;
pgnstr := "1. e4 e5 2. Nf3 Nc6 3. Nxe5 Nxe5 4. Bb5 c6";
build_chessboards_from_pgn(pgnstr);
beginfig(0);
init_backboard;
draw backboard;
show_last_move(3);
draw chessboard_step(3); % Nf3
draw_arrows(red)("f8--b4","g1|-f3");
endfig;
\end{ExempleMP}

The thickness of the arrows can be changed with the following command:
\commande|set_arrow_width(«coeff»)|\index{set_arrow_width@\lstinline+set_arrow_width+}\smallskip

\begin{description}
  \item[\meta{coeff}:] is a coefficient (\lstinline+numeric+) which allows you 
  to adjust the width of the arrows in proportion to the width of a square on the chessboard. By default, this coefficient is \lstinline+0.08+.
\end{description}

The following example illustrates the use of this command:

\begin{ExempleMP}
input mpchess;
string pgnstr;
pgnstr := "1. e4 e5 2. Nf3 Nc6 3. Nxe5 Nxe5 4. Bb5 c6";
build_chessboards_from_pgn(pgnstr);
beginfig(0);
set_black_view;
init_backboard;
draw backboard;
show_last_move(3);
draw chessboard_step(3); % Nf3
set_arrow_width(0.12);
draw_arrows(red)("f8--b4","g1|-f3");
endfig;
\end{ExempleMP}

\subsection{Coloring Squares}

\mpchess also allows you to color squares with the following command:

\commande|color_square(«color»)(«coord1»,«coord2», etc.)|\index{color_square@\lstinline+color_square+}\smallskip

\begin{description}
\item[\meta{color}:] is a \MP{} \lstinline+color+.
\item[\meta{coord1}:] is a string (between double quotes) consisting of two
coordinates (a letter and a number). 
\end{description}

The following example illustrates the use of this command:

\begin{ExempleMP}
input mpchess
beginfig(0);
init_backboard;
draw backboard;
color_square(red)("e2","e7","c5");
init_chessboard;
draw chessboard;
endfig;
\end{ExempleMP}

This command transparently colors the specified squares.

\subsection{Circles}

\mpchess allows you to surround squares with circles using the following command
 below:

\commande|draw_circles(«color»)(«coord1»,«coord2», etc.)|\index{draw_circles@\lstinline+draw_circles+}\smallskip

\begin{description}
\item[\meta{color}:] is a \MP{} \lstinline+color+.
\item[\meta{coord1}:] is a string (between double quotes) consisting of two
coordinates (letter and number). 
\end{description}

The following example illustrates the use of this command.

\begin{ExempleMP}
input mpchess
beginfig(0);
init_backboard;
draw backboard;
init_chessboard;
draw chessboard;
draw_circles(green)("e2","e7","c5");
endfig;
\end{ExempleMP}



\subsection{Crosses}


\mpchess allows you to draw crosses on squares with the following command:

\commande|draw_crosses(«color»)(«coord1»,«coord2», etc.)|\index{draw_crosses@\lstinline+draw_crosses+}\smallskip

\begin{description}
\item[\meta{color}:] is a \MP{} \lstinline+color+.
\item[\meta{coord1}:] is a string (between double quotes) consisting of two
coordinates (a letter and a number). 
\end{description}

The following example illustrates the use of this command.

  
\begin{ExempleMP}
input mpchess
beginfig(0);
init_backboard;
draw backboard;
init_chessboard;
draw chessboard;
draw_crosses(0.7[green,black])("e2","e7","c5");
endfig;
\end{ExempleMP}

\subsection{Move Comments}

\mpchess allows you to display the classic move comments of the type
``!?’’ with the following command:

\commande|draw_comment(«str»,«pos»)|\index{draw_comment@\lstinline+draw_comment+}\smallskip

\begin{description}
\item[\meta{str}:] is a string (between double quotes) to display.
\item[\meta{pos}:] is a string (between double quotes) consisting
of a coordinate (a letter and a number).
\end{description}

\begin{ExempleMP}
input mpchess;
string pgnstr;
pgnstr := "1. e4 e5 2. Nf3 Nc6 3. Nxe5 Nxe5 4. Bb5 c6";
build_chessboards_from_pgn(pgnstr);
beginfig(0);
init_backboard;
draw backboard;
show_last_move(3);
draw chessboard_step(3); % Nf3
draw_comment("?!","f3");
endfig;
\end{ExempleMP}

The color of the comment annotation can be changed with the following command:
\commande|set_comment_color(«color»)|\index{set_comment_color@\lstinline+set_comment_color+}\smallskip


\subsection{Main Lines}

\mpchess provides a command to display the arrows of the moves of the
main lines of analysis. There are commands for both colors. 

\commande|draw_white_main_lines(«move1»,«move2»,etc.)|\index{draw_white_main_lines@\lstinline+draw_white_main_lines+}\smallskip

\commande|draw_black_main_lines(«move1»,«move2»,etc.)|\index{draw_black_main_lines@\lstinline+draw_black_main_lines+}\smallskip

\begin{description}
  \item[\meta{move1}, \meta{move2}, etc.:] are the moves to be illustrated using
\textsc{pgn} notation.
\end{description}
  
When using the format \textsc{pgn} for the construction of the positions to be  
to be displayed, the following commands can be used  
to specify which move of the game is being commented on:

\commande|draw_white_main_lines_step(«step»)(«move1»,«move2»,etc.)|\index{draw_white_main_lines_step@\lstinline+draw_white_main_lines_step+}\smallskip

\commande|draw_black_main_lines_step(«step»)(«move1»,«move2»,etc.)|\index{draw_black_main_lines_step@\lstinline+draw_black_main_lines_step+}\smallskip

\begin{description}
\item[\meta{step}:] is the step of the game you want to annote; 
\item[\meta{move1}, \meta{move2}, etc.:] are the moves to be illustrated using
\textsc{pgn} notation.
\end{description}

The following example illustrates the use of this command:
\begin{ExempleMP}
input mpchess
string pgnstr;
pgnstr:="1. e4 d5";
build_chessboards_from_pgn(pgnstr);
beginfig(0);
init_backboard;
draw backboard;
draw chessboard_step(2);
draw_white_main_lines_step(2)("exd5","e5","Nc3");
endfig;
\end{ExempleMP}

To change the color (by default \lstinline+0.3[_blackColorSquare,black]+),
 use the following command:

\commande|set_main_lines_color(«color»)|\index{set_main_lines_color@\lstinline+set_main_lines_color+}\smallskip


\subsection{Possible Moves}\label{sec:possiblemoves}

\mpchess allows, by imitating \href{Lichess.org}{https://lichess.org},
to show possible moves for a piece. To do that, use the following command:

\commande|show_possible_moves(«square»)|\index{show_possible_moves@\lstinline+show_possible_moves+}\smallskip

\begin{description}
\item[\meta{square}:] is a string (between double quotes) consisting
of a coordinate (a letter and a number). 
\end{description}

This command must be used after having drawn the  \lstinline+chessboard+.

\begin{ExempleMP}
  input mpchess
  string fenstr;
  fenstr:="7r/2p1kp1p/p1B2p2/1pb5/8/2PP4/PP1N1PPP/R5K1 b - - 2 19";
  build_chessboard_from_fen(fenstr);
  beginfig(0);
  set_black_view;
  init_backboard;
  draw backboard;
  draw chessboard;
  show_possible_moves("c5");
  endfig;
\end{ExempleMP}  

In the case where several positions have been constructed with \textsc{pgn}
format, the following command should be used to access the chosen position. 

\commande|show_possible_moves_step(«step»)(«square»)|\index{show_possible_moves_step@\lstinline+show_possible_moves_step+}\smallskip

\begin{description}
  \item[\meta{step}:] is the step of the game you want to annote;
  \item[\meta{square}:] is a string (between double quotes) consisting
of a coordinate (a letter and a number).
\end{description}

By default, the color is set to \lstinline+0.4[green,black]+, but this can be changed with the can be changed with the following command:

\commande|set_possible_moves_color(«color»)|\index{set_possible_moves_color@\lstinline+set_possible_moves_color+}\smallskip

\section{Miscellaneous}

\subsection{Reset the \lstinline+chessboard+}

To reset the internal structure storing the positions of the parts,  use
the following command:


\commande|clear_chessboard|\index{clear_chessboard@\lstinline+clear_chessboard+}\smallskip


\subsection{Global Reset}

To reinitialize the values of the different parameters of \mpchess, use
the following command:

\commande|reset_mpchess|\index{reset_mpchess@\lstinline+reset_mpchess+}\smallskip


\subsection{Clip the board}

We can clip the chessboard with the following command.

\commande|clip_chessboard(«string»)|\index{clip_chessboard@\lstinline+clip_chessboard+}\smallskip


\begin{description}
\item[\meta{string}:] is a string (between double quotes)
composed of two coordinates (letter and number) separated by a dash; for example \lstinline+"a1-c6"+.
\end{description}

Here is an example of an illustration:
\begin{ExempleMP}
input mpchess;
string pgnstr;
pgnstr := "1. e4 e5 2. Nf3 Nc6 3. Nxe5 Nxe5 4. Bb5 c6";
build_chessboards_from_pgn(pgnstr);
beginfig(0);
set_black_view;
init_backboard;
draw backboard;
show_last_move(3);
draw chessboard_step(3); % Nf3
draw_comment("?!","f3");
clip_chessboard("e1-g4");
endfig;
\end{ExempleMP}


\section{Use with \LaTeX{}}

\subsection{Use with \hologo{pdfLaTeX} or \hologo{XeLaTeX}}

There are several ways to include the images produced by \mpchess in a \LaTeX
document. The first is to generate pdf files with \MP{} and then to include them
with \lstinline[language=TeX]+\includegraphics+. This solution works with all
engines.

You can also use the packages \package{gmp} or \package{mpgraphics} with
\hologo{pdfLaTeX} or \hologo{XeLaTeX}\footnote{We would like to thank Quark67
for the questions and advice}.

\subsubsection{With \package{mpgraphics}}

With \package{mpgraphics}~\cite{ctan-mpgraphics}, we load \mpchess with the
\lstinline+mpdefs+ environment and we can produce images with \MP{}
code but without using \lstinline+beginfig+ and \lstinline+endfig+;
rather, the code to generate a figure is in the
\lstinline+mpdisplay+ environment. It will be necessary to use the option
\lstinline+-shell-escape+ when compiling the \LaTeX document. 

Here is a complete example:

\begin{latexcode}
\documentclass{article}
\usepackage{mpgraphics}
\begin{document}
\begin{mpdefs}
input mpchess
\end{mpdefs}
\begin{mpdisplay}
init_backboard;
draw backboard;
init_chessboard;
draw chessboard;
draw_arrows(red)("e7--e5","g1|-f3");
\end{mpdisplay}
\end{document}
\end{latexcode}

\subsubsection{With \package{gmp}}

The use of the package \package{gmp}~\cite{ctan-gmp} is quite similar to that of
\package{mpgraphics}. Some commands are different, but as with
\package{mpgraphics}, we do not use \lstinline+beginfig+ and
\lstinline+endfig+. The loading of \mpchess can be done when loading the
package, and the \MP{} code is in a \lstinline+mpost+ environment. Here again
it will be necessary to compile the \LaTeX{} document with the
\lstinline+-shell-escape+ option. 


Here is a complete example:


\begin{latexcode}
\documentclass{article}
\usepackage[shellescape, everymp={input mpchess;}]{gmp}

\begin{document}

\begin{mpost}
init_backboard;
draw backboard;
init_chessboard;
draw chessboard;
draw_arrows(red)("e7--e5","g1|-f3");
\end{mpost}
\end{document}
\end{latexcode}
\subsection{Use with \hologo{LuaLaTeX} and \package{luamplib}}

It is possible to use \mpchess directly in a \LaTeX{} file with
\hologo{LuaLaTeX} and the package \package{luamplib}. (This is what is done to
write this documentation.)

It then suffices to put the \MP{} code in the environment 
\lstinline+mplibcode+.  

For certain functionalities, \mpchess uses the \MP{} operator
\lstinline+infont+. Thus, in order for the content of these features to be composed in the current font of the document, one must add the command
\begin{latexcode}
\mplibtextextlabel{enable}
\end{latexcode}



For more details on these mechanisms, refer to the
+\package{luamplib}~\cite{ctan-luamplib} documentation.

We can load globally \mpchess with the following command:

\begin{latexcode}
\everymplib{input mpchess;}
\end{latexcode}

Here is a complete example (to be compiled with
\hologo{LuaLaTeX}). 


\begin{latexcode}
\documentclass{article}
\usepackage{luamplib}

\everymplib{input mpchess;}

\begin{document}

\begin{mplibcode}
beginfig(0);
init_backboard;
draw backboard;
init_chessboard;
draw chessboard;
draw_arrows(red)("e7--e5","g1|-f3");
endfig;
\end{mplibcode}
\end{document}
\end{latexcode}

\subsection{TrueType Font}

The \mpchess{} package provides a very simple font composed of only the 12
glyphs corresponding to the black and white pieces in the Unicode table. To
access it you just have to use \hologo{LuaLaTeX} or \hologo{XeLaTeX} and the
package \package{fontspec}.  To facilitate its use, we propose
some definitions.

\begin{latexcode}
  \documentclass{article}
  \usepackage{fontspec}
  \newfontfamily{\chessfont}{mpchess font}
  \newcommand\bP{{\chessfont \char"265F}} % black Pawn
  \newcommand\bN{{\chessfont \char"265E}} % black Knight
  \newcommand\bB{{\chessfont \char"265D}} % black Bishop
  \newcommand\bR{{\chessfont \char"265C}} % black Rook
  \newcommand\bQ{{\chessfont \char"265B}} % black Queen
  \newcommand\bK{{\chessfont \char"265A}} % black King
  \newcommand\wP{{\chessfont \char"2659}} % white Pawn
  \newcommand\wN{{\chessfont \char"2658}} % white Knight
  \newcommand\wB{{\chessfont \char"2657}} % white Bishop
  \newcommand\wR{{\chessfont \char"2656}} % white Rook
  \newcommand\wQ{{\chessfont \char"2655}} % white Queen
  \newcommand\wK{{\chessfont \char"2654}} % white King
  
  \begin{document}
  
 1. e4 e5 2. \wB c4 d6 3. \wN f3 \bB g4 4. \wN
  c3 g6 5. \wN xe5 \bB xd1.
  
  \end{document}
  \end{latexcode}
  
  This code will produce:
  
  "1. e4 e5 2. \wB c4 d6 3. \wN f3 \bB g4
  4. \wN c3 g6 5. \wN xe5 \bB xd1."

\section{To Do}

Many things can be added to \mpchess. Among these, we can think of:
\begin{itemize}
\item display the captured pieces, or the differential of the captured pieces;
\item display the remaining time for each player; 
\item parsing the complete \textsc{pgn} format with the optional tags ;
\item adding the coordinates outside the board when it is clipped;
\item add piece themes.
\end{itemize}

\section{Complete Example}

\begin{ExempleMP}[sidebyside=false]
input mpchess
string pgnstr;
pgnstr:="1. e4 e5 2. Bc4 d6 3. Nf3 Bg4 4. Nc3 g6 5. Nxe5 Bxd1";
build_chessboards_from_pgn(pgnstr);
beginfig(0);
set_white_player("Kermur de Legal");
set_black_player("Saint-Brie");
set_backboard_width(8cm);
init_backboard;
draw backboard;
show_last_move(10);
draw_comment("?","d1");
color_square(0.3[green,black])("c4","c3","e5");
color_square(0.3[red,black])("e8");
draw chessboard_step(10);
draw_arrows(0.3[green,black])("e5|-f7","c3-|d5");
draw_arrows(0.3[red,black])("c4--f7");
endfig;
\end{ExempleMP}

\section{History}

\begin{description}
\item[v0.7, juillet 2023:] Black knight adjustment, adding color board themes (\lstinline+GreenLichess+, \lstinline+PinkPyramidalLichess+,
\lstinline+Wood+ with color type (\lstinline+set_board_type+).
\item[v0.6, April 26, 2023:] Fixed bugs concerning castling management, and
 moves ambiguities in \textsc{pgn} format.
Changed \lstinline+set_white_color+ to
\lstinline+set_white_squares_color+ and \lstinline+set_black_color+ to
\lstinline+set_black_squares_color+. Change \lstinline+set_no_coords+ to
\lstinline+hide_coordinates+ and \lstinline+set_coords+ to
\lstinline+show_coordinates+. Change \lstinline+set_whos_to_move+ to
\lstinline+show_whos_to_move+ and \lstinline+unset_whos_to_move+ to
\lstinline+hide_whos_to_move+. Change \lstinline+set_coords_inside+ and
\lstinline+set_coords_outside+ to \lstinline+set_coordinates_position+.
Change \lstinline+set_coords_font+ to \lstinline+set_coordinates_font+. 

Translated with www.DeepL.com/Translator (free version)
\item[v0.5, April 20, 2023:]  Bug fixed, \textbf{changing the
default piece set} for the \lstinline+mpchess+ set (which has been added to Lichess), added TrueType font, and updated documentation.
\item[v0.4, April 6, 2023:] Corrections in the documentation, especially the
English version; added commands to visualize the possible movements for a
 piece (section~\ref{sec:possiblemoves}).
\item[v0.3, March 29, 2023:] Small bug. 
\item [v0.2, March 28, 2023:] Added commands for \textsc{pgn} and \textsc{fen}
file reading; added commands for displaying the main lines of
analysis; removal of the \lstinline+staunty+ theme (because of
license) and creation of the \lstinline+mpchess+ parts theme.
\item[v0.1, March 23, 2023:] First publication on the \ctan.
\end{description}

\section{Acknowledgements}

The author would like to thank Quark67 for his feedback and corrections, 
Douglas Johnson for correcting the English version of the documentation and Hans
Nieuwenhuis for his advices. 
This
feedback and encouragement is extremely appreciated!


\printbibliography
\printindex
\end{document}



%%% Local Variables:
%%% flyspell-mode: 1
%%% ispell-local-dictionary: "fr"
%%% End:
