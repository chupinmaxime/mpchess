
% LTeX: language=en

% mpchess: draw chess boards and positions with MetaPost
%
% Originally written by Maxime Chupin <notezik@gmail.com>,
% 2023.
%
% Distributed under the terms of the GNU free documentation licence:
%   http://www.gnu.org/licenses/fdl.html
% without any invariant section or cover text.

\documentclass[english]{ltxdoc}
\usepackage{tcolorbox}
\tcbuselibrary{listings,breakable}
%\tcbuselibrary{documentation}
\usepackage{enumitem}
\usepackage[tikz]{bclogo}
\usepackage{mflogo}
\usepackage{hologo}
\usepackage{luamplib}
\mplibtextextlabel{enable}
\usepackage{biblatex}
\addbibresource{ctan.bib}
\usepackage{wrapfig}
\usepackage{siunitx}
\usepackage{imakeidx}
\usepackage{fancyvrb,xparse,xargs}
\usepackage[sfdefault]{FiraSans}
\usepackage[mathrm=sym]{firamath-otf}
%\setmonofont{Fira Mono}
\setmonofont{FiraCode-Regular.ttf}[BoldFont= FiraCode-Bold.ttf,ItalicFont= FiraCode-RegularItalic.otf,BoldItalicFont= FiraCode-BoldItalic.otf,]

\usepackage{xspace}
\usepackage{animate}
\usepackage[english]{babel}
\newcommand{\ctan}{\textsc{ctan}}
\NewDocumentCommand{\package}{ m }{%
  \href{https://ctan.org/pkg/#1}{#1}\xspace
}

\definecolor{darkred}{rgb}{0.6,0.1,0.1}
\definecolor{vert}{rgb}{0.1,0.4,0.1}
\definecolor{bleu}{rgb}{0.2,0.2,0.6}
\definecolor{orange}{rgb}{0.6,0.4,0.}
\colorlet{code}{blue!80!black}

\usepackage[colorlinks=true,urlcolor=orange,linkcolor=orange,menucolor=black,citecolor=orange]{hyperref}

\newcommand \file       {\nolinkurl}
\renewcommand \cmd        {\texttt}
\renewcommand \code   [1] {\texorpdfstring {\texttt{\color{code}#1}} {#1}}
\renewcommand*\cs     [1] {\code{\textbackslash #1}}



\newcommand*\commande{\par\bigskip\noindent\hspace{-30pt}%
  \SaveVerb[aftersave={%
   \UseVerb{Vitem}%
  }%
  ]{Vitem}%
  }
  \newcommand\vitem[1][]{\SaveVerb[%
  aftersave={\item[\textnormal{\UseVerb[#1]{vsave}}]}]{vsave}}
\newcommand*\textme[1]{\textcolor{black}{\rmfamily\textit{#1}}}
%\renewcommand*\meta[1]{% % meta
%  \textme{\ensuremath{\langle}#1\ensuremath{\rangle}}}
\newcommand*\optstar{% % optional star
  \meta{\ensuremath{*}}\xspace}
\DefineShortVerb{\|}
\newcommand\R{\mathbf{R}}
\setlength{\fboxsep}{2pt}
\fvset{%
  codes={\catcode`\«\active \catcode`\×\active },
  defineactive={\makefancyog\makefancytimes},
  formatcom=\color{darkred},
  frame=single
}
% rendre «...» équivalent à \meta{...}
{\catcode`\«\active
  \newcommandx\makefancyog[0][addprefix=\global]{%
    \def«##1»{\meta{##1}}}}
% rendre × équivalent à \optstar
{\catcode`\×\active
  \newcommandx\makefancytimes[0][addprefix=\global]{%
    \def×{\optstar{}}}}


\newcommand\mpchess{\textbf{\textlogo{MP}}\textit{chess}\xspace}



%\addbibresource{biblio.bib}


\lstset{
  numberstyle=\footnotesize\color{vert},
  keywordstyle=\ttfamily\bfseries\color{bleu},
  basicstyle=\ttfamily,
  commentstyle=\itshape\color{vert},
  stringstyle=\ttfamily\color{orange},
  showstringspaces=false,
  language=MetaPost,
  breaklines=true,
  breakindent=30pt,
  defaultdialect=MetaPost,
  classoffset=1,% frame=tb
  morekeywords={init_backboard,set_backboard_width,set_backboard_size,set_color_theme,get_backboard_width,get_backboard_size,set_backboard_width,get_square_dim,set_white_color,set_black_color,set_coords_inside,set_coords_outside,set_coords_font,set_coords,
  set_no_coords,set_white_view,set_black_view,
  set_white_player,set_black_player,set_pieces_theme,set_players_side,init_chessboard,set_empty_chessboard,add_white_pieces,add_black_pieces,clear_areas,clear_files,clear_ranks,clear_squares,build_chessboard_from_fen,build_chessboards_from_pgn,clear_chessboard,
  color_square,draw_arrows,draw_circles,draw_comment,draw_crosses,get_halfmove_number,get_totalmove_number,
  reset_mpchess,set_black_to_move,set_last_move_color,set_comment_color,set_white_to_move,set_whos_to_move,show_last_move,unset_whos_to_move,set_arrow_width,clip_chessboard},
  keywordstyle=\color{darkred},
  classoffset=2,% frame=tb
  morekeywords={backboard,chessboard,chessboard_step},
  keywordstyle=\color{vert},
  classoffset=0,% frame=tb
  morekeywords={draw},
  keywordstyle=\color{bleu}
}

\makeatletter
\tcbset{%
    listing metapost/.code={%
        \def\tcbuselistingtext@input{\begin{mplibcode} input \jobname.listing; \end{mplibcode}}%
    }
}
\makeatother
\newtcblisting[auto counter,]{ExempleMP}[1][]{%
  arc=0pt,outer arc=0pt,
  colback=darkred!3,
  colframe=darkred,
  breakable,fontupper=\small,
  boxsep=0pt,left=5pt,right=5pt,top=5pt,bottom=5pt, bottomtitle =
  3pt, toptitle=3pt,
  boxrule=0pt,bottomrule=0.5pt,toprule=0.5pt, toprule at break =
  0pt, bottomrule at break = 0pt,
  listing side text,
  listing metapost,
  title=Exemple~\thetcbcounter,
  listing options={breaklines},#1
}

\newtcblisting{commandshell}{colback=black,colupper=white,colframe=black,
  arc=0pt,
  listing only,boxsep=0pt,listing
  options={style=tcblatex,language=sh},
  every listing line={\textcolor{red}{\small\ttfamily\bfseries user \$> }}}


  \newtcblisting{mpcode}{
  arc=0pt,outer arc=0pt,
  colback=darkred!3,
  colframe=darkred,
  breakable,
  boxsep=0pt,left=5pt,right=5pt,top=5pt,bottom=5pt, bottomtitle =
  3pt, toptitle=3pt,
  boxrule=0pt,bottomrule=0.5pt,toprule=0.5pt, toprule at break =
  0pt, bottomrule at break = 0pt,
  listing only,boxsep=0pt,listing
  options={breaklines}
}

\newtcblisting{latexcode}{
  arc=0pt,outer arc=0pt,
  colback=darkred!3,
  colframe=darkred,
  breakable,
  boxsep=0pt,left=5pt,right=5pt,top=5pt,bottom=5pt, bottomtitle =
  3pt, toptitle=3pt,
  boxrule=0pt,bottomrule=0.5pt,toprule=0.5pt, toprule at break =
  0pt, bottomrule at break = 0pt,
  listing only,boxsep=0pt,listing
  options={breaklines,language={[LaTeX]TeX}}
}


\makeindex[title=Command Index, columns=2]



%\lstset{moredelim=*[s][\color{red}\rmfamily\itshape]{<}{>}}
%\lstset{moredelim=*[s][\color{blue}\rmfamily\itshape]{<<}{>>}}

\begin{document}

\title{{MPchess} : drawing chess boards and positions with \hologo{METAPOST}}
\author{Maxime Chupin, \url{notezik@gmail.com}}
\date{\today}

%% === Page de garde ===================================================
\thispagestyle{empty}
\begin{tikzpicture}[remember picture, overlay]
  \node[below right, shift={(-4pt,4pt)}] at (current page.north west) {%
    \includegraphics{fond.pdf}%
  };
\end{tikzpicture}%

\noindent
{\Huge \mpchess}\par\medskip
\noindent
{\Large  drawing chess boards and positions with \hologo{METAPOST}}\\[1cm]
\parbox{0.6\textwidth}{
  \begin{mplibcode}
    input mpchess
    string pgnstr;
    pgnstr:="1. e4 e5 2. Bc4 d6 3. Nf3 Bg4 4. Nc3 g6 5. Nxe5 Bxd1 ";
    build_chessboards_from_pgn(pgnstr);
    beginfig(0);
    set_backboard_width(8cm);
    init_backboard;
    draw backboard;
    show_last_move(10);
    draw_comment("?","d1");
    color_square(0.3[green,black])("c4","c3","e5");
    color_square(0.3[red,black])("e8");
    draw chessboard_step(10);
    draw_arrows(0.3[green,black])("e5|-f7","c3-|d5");
    draw_arrows(0.3[red,black])("c4--f7");
    endfig;
  \end{mplibcode}
}\hfill
\parbox{0.5\textwidth}{\Large\raggedleft
  \textbf{Contributor}\\
  Maxime \textsc{Chupin}\\
  \url{notezik@gmail.com}
}
\vfill
\begin{center}
  Version 0.1, 2023, March, 23th \\
  \url{https://plmlab.math.cnrs.fr/mchupin/mpchess}
\end{center}
%% == Page de garde ====================================================
\newpage

%\maketitle

\begin{abstract}
  This package allows you to draw chess boards and positions.
  The appearance of the drawings is modern and largely inspired by what is offered by
  the excellent web site \url{Lichess.org}.
  Relying on \MP{} probably allows more graphic flexibility than the
  excellent packages \LaTeX{}.
\end{abstract}


\begin{center}
  \url{https://plmlab.math.cnrs.fr/mchupin/mpchess}
\end{center}

\tableofcontents

\begin{itshape}
  This package is in beta version, do not hesitate to report bugs, as well as requests for improvement.
\end{itshape}

\section{Installation}

\mpchess is on the \ctan{} and can be installed via the package manager of your
distribution.

\begin{center}
  \url{https://www.ctan.org/pkg/mpchess}
\end{center}


\subsection{With \TeX live under Linux or MacOS}

To install \mpchess with \TeX live, you will have to create the directory
\lstinline+texmf+ directory in your \lstinline+home+. 

\begin{commandshell}
mkdir ~/texmf
\end{commandshell}

Then, you will have to place the \lstinline+.mp+ files in the 
\begin{center}
  \lstinline+~/texmf/tex/metapost/mpchess/+
\end{center}

\mpchess consists of 7 files \hologo{METAPOST} :
\begin{itemize}
  \item \verb+mpchess.mp+;
  \item \verb+mpchess-chessboard.mp+;
  \item \verb+mpchess-pgn.mp+;
  \item \verb+mpchess-fen.mp+;
  \item \verb+mpchess-cburnett.mp+;
  \item \verb+mpchess-staunty.mp+;
  \item \verb+mpchess-skak.mp+.
\end{itemize}

Once this is done, \mpchess will be loaded with the classic
\begin{mpcode}
input mpchess
\end{mpcode}

\subsection{With Mik\TeX{} and Windows}

These two systems are unknown to the author of \mpchess, so we refer to their documentation to add local packages:
\begin{center}
  \url{http://docs.miktex.org/manual/localadditions.html}
\end{center}



\subsection{Dependencies}


\mpchess depends on the packages \MP: \package{hatching} and, if
\mpchess is not used with \hologo{LuaLaTeX} and package{luamplib},
\package{latexmp}.

\subsection{Use with \hologo{LuaLaTeX} and \package{luamplib}}

It is possible to use \mpchess directly in a \LaTeX{} file with
\hologo{LuaLaTeX} and the package \package{luamplib}. This is what is done to
write this documentation. 

For certain functionalities, \mpchess uses the \MP{} operator
\lstinline+infont+. Thus, in order for the content of these features to be composed in the current font of the document, one must add the command :
\begin{latexcode}
\mplibtextextlabel{enable}
\end{latexcode}

For more details on these mechanisms, we refer to the documentation of the
package \package{luamplib}~\cite{ctan-luamplib}.

\section{Why this package and general philosophy}

There are already \LaTeX{} packages for drawing chess boards and positions,
including the very good \package{xskak}~\cite{ctan-xskak}coupled with the
package \package{chessboard}~\cite{ctan-chessboard}. 
Ulrike Fisher has done there improvement, maintenance work, and provided us with
excellent tools to make chess diagrams and to handle the different formats of
game descriptions\footnote{She even developed the
package to handle various chess fonts.}. The
documentations of these packages are very good.

Several things motivated the creation of \mpchess. First of all, with
\package{chessboard} the addition of a set of pieces is not very easy, because
it relies on fonts. Moreover, I find that drawing chess game diagrams is
something diagrams is a very graphical thing, and that using a dedicated drawing
language offers more flexibility and what better than \MP~\cite{ctan-metapost}.

With \mpchess, we build the final image of the chess board with the pieces
by successive layers. Thus, we begin by producing and drawing the board
(\lstinline+backboard+), which we can modify by coloring some squares for
example, then we will add the pieces of the position (\lstinline+chessboard+),
and finally, we can annotate the whole with marks, colors, arrows, etc. 



Moreover, \mpchess produces images that are graphically close to what can be
provided by excellent  \emph{open source} website \url{https://lichess.org}. You
will see that the colors, the pieces and the general aspect are largely inspired
of what this website offers.

\section{Board}

The board is called with \mpchess{} \lstinline+backboard+.
You have to initialize the board before drawing it. This is done with the
following command:\par

\commande|init_backboard|\smallskip\index{init_backboard@\lstinline+init_backboard+}

This command constructs a \MP{} \lstinline+picture+ named
\mbox{\lstinline+backboard+.} It should then be drawn as illustrated in the following example.
\begin{ExempleMP}
input mpchess

beginfig(0);
init_backboard;
draw backboard;
endfig;
\end{ExempleMP}

This initialization will make it possible to take into account the various options and features that we will describe in the following.

\subsection{Setting the size}

When creating the \lstinline+backboard+, you can decide on the width of it. This is done with the following command:\par\bigskip

\commande|set_backboard_width(«dim»)|\smallskip\index{set_backboard_width@\lstinline+set_backboard_width+}

\begin{description}
  \item[\meta{dim}:] is the width of the board (with units). By default, this dimension is \SI{5}{cm}.
\end{description}

The use of this command is illustrated in the example~\ref{ex:widthcase}. This
command should be used before \lstinline+init_backboard+ so that it is taken
into account when creating the \lstinline+picture+. 
\bigskip

The size of the board can be retrieved with the following command:

\commande|get_backboard_width|\smallskip\index{get_backboard_width@\lstinline+get_backboard_width+}

This command returns a \lstinline+numeric+.

\subsection{Number of squares}

By default, the game board contains 64 squares ($8\times 8$). You can change
this with the following command: 
\commande|set_backboard_size(«nbr»)|
\smallskip\index{set_backboard_size@\lstinline+set_backboard_size+}

\begin{description}
  \item[\meta{nbr}:] is the desired number of squares. The board will then be
  square of size \meta{nbr}$\times$\meta{nbr}. By default this number is 8. 
\end{description}

Again, this command must be used before \lstinline+init_backboard+ for it to be
taken into account as shown in the following example. 

\begin{ExempleMP}[label=ex:widthcase]
input mpchess

beginfig(0);
set_backboard_width(3cm);
set_backboard_size(6);
init_backboard;
draw backboard;
endfig;
\end{ExempleMP}

To obtain the size of the game board, you can use the following command the
following command 

\commande|get_backboard_size|\smallskip\index{get_backboard_size@\lstinline+get_backboard_size+}

This command returns a \lstinline+numeric+.

\subsection{Dimension of a square}

Depending on the number of squares on the board and the prescribed width of the board, \mpchess calculates the dimension (width or height) of a square. This
serves as a general unit. To obtain it, use the following command.

\commande|get_square_dim|\smallskip\index{get_square_dim@\lstinline+get_square_dim+}

This command returns a \lstinline+numeric+.

\subsection{Setting the color theme}

\subsubsection{Predefined themes}

Several color themes are available. To choose a color theme, use the following
command:

\commande|set_color_theme(«string»)|\smallskip\index{set_color_theme@\lstinline+set_color_theme+}

\begin{description}
  \item[\meta{string}] can be:
\begin{itemize}
\item \lstinline+"BlueLichess"+ (thème par défaut);
\item \lstinline+"BrownLichess"+ ;
\item or \lstinline+"Classical"+.
\end{itemize}
\end{description}

The following examples show the results obtained.
\begin{ExempleMP}
input mpchess
beginfig(0);
set_color_theme("BrownLichess");
init_backboard;
draw backboard;
endfig;
\end{ExempleMP}
\begin{ExempleMP}
input mpchess
beginfig(0);
set_color_theme("Classical");
init_backboard;
draw backboard;
endfig;
\end{ExempleMP}

The two color themes provided borrow the colors of the Lichess themes.


\subsubsection{Configuration of a personal theme}

A color theme is really just the definition of two colors.
These can be defined with the following commands.

\commande|set_white_color(«color»)|\index{set_white_color@\lstinline+set_white_color+}\par
\commande|set_black_color(«color»)|\index{set_black_color@\lstinline+set_black_color+}\smallskip

\meta{color} is a \MP{} \lstinline+color+.

\begin{ExempleMP}
input mpchess
beginfig(0);
set_white_color((0.9,0.8,0.8));
set_black_color((0.7,0.6,0.6));
init_backboard;
draw backboard;
endfig;
\end{ExempleMP}
\subsection{Display coordinates}

You may have noticed in the various examples that by default, the coordinates are, as on the Lichess site, written in small letters inside the boxes.

\mpchess allows to position these coordinates outside the board with the
following command. 

\commande|set_coords_outside|\index{set_coords_outside@\lstinline+set_coords_outside+}\smallskip

The result is as follows.



\begin{ExempleMP}
input mpchess
beginfig(0);
set_coords_outside;
init_backboard;
draw backboard;
endfig;
\end{ExempleMP}

There is also a command to position the coordinates inside the board.

\commande|set_coords_inside|\index{set_coords_inside@\lstinline+set_coords_inside+}\smallskip

You can see in this documentation that with \package{luamplib} and
\LaTeX, the font is the font of the current document. To draw these letters and
these numbers, \mpchess uses the \MP{} operator  \lstinline+infont+ and the font
is set to \lstinline+defaultfont+ by default\footnote{With \package{luamplib}
the \lstinline+infont+ operator is redefined and its argument is simply
ignored}, so it is not possible to modify the composition font of the
coordinates. This font can be changed with the following command.

\commande|set_coords_font(«font»)|\index{set_coords_font@\lstinline+set_coords_font+}\smallskip

It will then be necessary to use the naming conventions specific to the \MP{}
operator
\lstinline+infont+ and we refer to the
documentation~\cite{ctan-metapost} for more details.

You can also delete the coordinates with the following command.

\commande|set_no_coords|\index{set_no_coords@\lstinline+set_no_coords+}\smallskip

And the reverse command also exists.

\commande|set_coords|\index{set_coords@\lstinline+set_coords+}\smallskip


\subsection{White or black side}

To choose if you want to see the tray on the white or black side, \mpchess
provides two commands. 

\commande|set_white_view|\index{set_white_view@\lstinline+set_white_view+}\smallskip

\commande|set_black_view|\index{set_black_view@\lstinline+set_black_view+}\smallskip

By default, we see the board on the white side.

\subsection{Players' names}

You can fill in the names of the players so that they are noted around the
chessboard. This is done with the following commands. 

\commande|set_white_player(«string»)|\index{set_white_player@\lstinline+set_white_player+}\smallskip

\commande|set_black_player(«string»)|\index{set_black_player@\lstinline+set_black_player+}\smallskip

\begin{description}
  \item[\meta{string} :] is the string interpreted by \LaTeX to display.
\end{description}

\begin{ExempleMP}
input mpchess
beginfig(0);
set_white_player("\textbf{GM} Kasparov");
set_black_player("\textbf{GM} Kramnik");

init_backboard;
draw backboard;
endfig;
\end{ExempleMP}

It is possible to place the names on the right side of the board without the
black bands present by default. 
This happens either if the coordinates are printed outside the tray, or if the
following command is used.

\commande|set_players_side|\index{set_players_side@\lstinline+set_players_side+}\smallskip

\begin{ExempleMP}[righthand width=0.6\linewidth]
input mpchess
beginfig(0);
set_white_player("\textbf{GM} Kasparov");
set_black_player("\textbf{GM} Kramnik");
set_players_side;
init_backboard;
draw backboard;
endfig;
\end{ExempleMP}



\section{Pieces et positions}

\mpchess, as described above, builds the picture of a chess position
layer by layer. This part is dedicated to the configuration of pieces and
positions.

Internally, \mpchess builds a table on the board grid. Then, some
macros allow to generate a \lstinline+picture+ to be drawn \emph{over} the board (\lstinline+backboard+).

\subsection{Setting the theme of the pieces}

\mpchess provides for the moment three themes of pieces, two borrowed from
Lichess, and the other borrowed from the
package~\package{skak}~cite{ctan-skak}\footnote{Which provides the \MF{} code
for a chess font, which has been easily adapted to \MP{} for \mpchess.}.

To choose the theme we will use the following command.

\commande|set_pieces_theme(«string»)|\index{set_pieces_theme@\lstinline+set_pieces_theme+}\smallskip

\begin{description}
  \item[\meta{string}:] can be:
\begin{itemize}
\item \lstinline+"cburnett"+ (default), to get the Lichess \emph{cburnett}
pieces set;
\item \lstinline+"staunty"+, to get the Lichess \emph{staunty}
pieces set;
\item \lstinline+"skak"+, to get the \package{skak}
pieces set.
\end{itemize}
\end{description}


The table~\ref{tab:pieces} shows the result of the three sets of pieces.
\begin{table}
  \centering
\begin{tabular}{cc}

\lstinline+cburnett+ theme&\lstinline+staunty+ theme\\
\begin{mplibcode}
  input mpchess
  beginfig(0);
  init_chessboard;
  set_backboard_width(4cm);
  set_pieces_theme("cburnett");
  unset_whos_to_move;
  init_backboard;
  draw backboard;
  draw chessboard_step(0);
  endfig;
\end{mplibcode}
&\begin{mplibcode}
  input mpchess
  beginfig(0);
  init_chessboard;
  set_backboard_width(4cm);
  set_pieces_theme("staunty");
  unset_whos_to_move;
  init_backboard;
  draw backboard;
  draw chessboard_step(0);
  endfig;
\end{mplibcode}\\
\lstinline+skak+ theme&\\
\begin{mplibcode}
  input mpchess
  beginfig(0);
  init_chessboard;
  set_backboard_width(4cm);
  set_pieces_theme("skak");
  unset_whos_to_move;
  init_backboard;
  draw backboard;
  draw chessboard_step(0);
  endfig;
\end{mplibcode}&\\
\end{tabular}
\caption{The different themes of pieces provided by \mpchess.}\label{tab:pieces}
\end{table}


\subsection{Who’s to move}

\mpchess indicates who has the move between white and black. This is done by
a small colored triangle (white or black) at the end of the board (which you can
you can see in the following examples).

To specify who has to move we use the following commands.

\commande|set_white_to_move|\index{set_white_to_move@\lstinline+set_white_to_move+}\smallskip


\commande|set_black_to_move|\index{set_black_to_move@\lstinline+set_black_to_move+}\smallskip

By default, white has to move, and this information is displayed.

To activate or deactivate this display, use one of the following two
commands.

\commande|set_whos_to_move|\index{set_whos_to_move@\lstinline+set_whos_to_move+}\smallskip


\commande|unset_whos_to_move|\index{unset_whos_to_move@\lstinline+unset_whos_to_move+}\smallskip

\subsection{Draw position}

The commands described below allow you to build a position in several ways
(adding pieces one by one, reading a \textsc{fen} file, etc.). Once a position
has been constructed, it can be plotted 
using the following command which generates a \MP{} \lstinline+picture+.


\commande|chessboard|\index{chessboard@\lstinline+chessboard+}\smallskip \label{com:chessboard}

The use of this command will be widely illustrated in the following examples.

\subsection{Build a position}

\subsubsection{Initialization}

To obtain the initial position of a game, simply use the following command.

\commande|init_chessboard|\index{init_chessboard@\lstinline+init_chessboard+}\smallskip

\begin{ExempleMP}
input mpchess
beginfig(0);
init_backboard;
draw backboard;
init_chessboard;
draw chessboard;
endfig;
\end{ExempleMP}

You can also initialize an empty \lstinline+chessboard+ with the following command. 

\commande|set_empty_chessboard|\index{set_empty_chessboard@\lstinline+set_empty_chessboard+}\smallskip

\subsubsection{Adding pieces}

You can add pieces to build a position with the following two commands.

\commande|add_white_pieces(«piece1»,«piece2»,etc.)|\index{add_white_pieces@\lstinline+add_white_pieces+}\smallskip


\commande|add_black_pieces(«piece1»,«piece2»,etc.)|\index{add_black_pieces@\lstinline+add_black_pieces+}\smallskip

These commands take lists of \textbf{\meta{piece}} which are strings
that describe the piece and the position using the
algebraic notation. There is no limitation on the number of pieces in the list.

The following example illustrates the use of these commands.

\begin{ExempleMP}
input mpchess
beginfig(0);
init_backboard;
draw backboard;
set_empty_chessboard;
add_white_pieces("e1","Kd2");
add_black_pieces("e7","f6","Kb8");
draw chessboard;
endfig;
\end{ExempleMP}


\subsubsection{Delete piece}

\mpchess provides several commands to remove items from a position.

The first command allows you to delete an item from a square. This command takes
a list of squares, using algebraic notation.

\commande|clear_squares(«square1»,«square2»,etc.)|\index{clear_squares@\lstinline+clear_squares+}\smallskip

The \textbf{\meta{square1}}, \textbf{\meta{square2}}, etc., are strings, for example \lstinline+"a3"+.
\medskip

The following command allows to delete a set of squares in a region
determined by two coordinates on the board. This command allows to take
a list of regions.

\commande|clear_areas(«area1»,«area2»,etc.)|\index{clear_areas@\lstinline+clear_areas+}\smallskip

The \textbf{\meta{area1}}, \textbf{\meta{area2}}, etc., are strings consisting of two coordinates separated by a dash, for example \lstinline+"a3-g7"+.
\medskip

The following command deletes all the cells in a file (column) determined by a
letter on the board. This command allows to take a list of files.


\commande|clear_files(«file1»,«file2»,etc.)|\index{clear_files@\lstinline+clear_files+}\smallskip

The \textbf{\meta{file1}}, \textbf{\meta{file2}}, etc., are strings of characters consisting of a letter, for example \lstinline+"a"+.
\medskip

The following command deletes all the cells in a rank (line) determined by a
number on the board. This command allows to take a list of ranks.

\commande|clear_ranks(«rank1»,«rank2»,etc.)|\index{clear_ranks@\lstinline+clear_ranks+}\smallskip

The \textbf{\meta{rank1}}, \textbf{\meta{rank2}}, etc., are strings made up of a number, for example \lstinline+"4"+.

The use of all these commands is illustrated in the following example.

\begin{ExempleMP}
input mpchess
beginfig(0);
init_backboard;
draw backboard;
init_chessboard;
clear_squares("a1","b2");
clear_areas("c2-d7");
clear_files("f","h");
clear_ranks("8");
draw chessboard;
endfig;
\end{ExempleMP}



\subsection{Reading data in \textsc{fen} format}

\mpchess allows you to read a position in the \textsc{fen} format thanks to the following command.

\commande|build_chessboard_from_fen(«string»)|\index{build_chessboard_from_fen@\lstinline+build_chessboard_from_fen+}\smallskip

\begin{description}
\item[\meta{string}:] is a string describing a position in
\textsc{fen} format. Note that all information after the 
\texttt{w} or \texttt{b} character is ignored. 
\end{description}

\begin{ExempleMP}
input mpchess;
beginfig(0);
init_backboard;
draw backboard;
string fenstr;
fenstr := "7r/2p1kp1p/p1B2p2/1pb5/8/2PP4/PP1N1PPP/R5K1 b - - 2 19";
build_chessboard_from_fen(fenstr);
draw chessboard;
endfig;
\end{ExempleMP}


\subsection{Reading data in \textsc{pgn} format}

\mpchess also makes it possible to read a string in the \textsc{pgn}
format. 
When such a functionality is used, \mpchess stores all the
intermediate positions and thus allows to represent them.

To construct the positions, we use the following command.

\commande|build_chessboards_from_pgn(«string»)|\index{build_chessboards_from_pgn@\lstinline+build_chessboards_from_pgn+}\smallskip

The \textsc{pgn} format accepted is a simplified one which does not accept variants or comments.

Once the positions are built, we can represent them with the following command.

\commande|chessboard_step(«int»)|\index{chessboard_step@\lstinline+chessboard_step+}\smallskip
\begin{description}
\item[\meta{int}:] is the number of the step. The initial configuration is numbered 0, and then each move, white or black, is numbered.
\end{description}

This command, like \lstinline+chessboard+ (see page~\pageref{com:chessboard}),
returns a \lstinline+picture+. 

The following example illustrates the use of these commands.

\begin{ExempleMP}
input mpchess;
string pgnstr;
pgnstr := "1. e4 e5 2. Nf3 Nc6 3. Nxe5 Nxe5 4. Bb5 c6";
build_chessboards_from_pgn(pgnstr);
beginfig(0);
init_backboard;
draw backboard;
draw chessboard_step(3); % Nf3
endfig;
\end{ExempleMP}

\subsubsection{Show last move}

The last move can be displayed automatically with the following command.

\commande|show_last_move(«int»)|\index{show_last_move@\lstinline+show_last_move+}\smallskip

\begin{description}
\item[\meta{int}:] is the number of the step. The initial setup is numbered 0, and then each move, white or black, is numbered.
\end{description}

This command colors in transparency the two squares where the last move starts
and ends. Thus, it must be used between the drawing of the board
(\lstinline+draw backboard+) and the drawing of the pieces 
(\lstinline+draw chessboard_step(i)+).

\begin{ExempleMP}
input mpchess;
string pgnstr;
pgnstr := "1. e4 e5 2. Nf3 Nc6 3. Nxe5 Nxe5 4. Bb5 c6";
build_chessboards_from_pgn(pgnstr);
beginfig(0);
init_backboard;
draw backboard;
show_last_move(3);
draw chessboard_step(3); % Nf3
endfig;
\end{ExempleMP}

You can configure the color used to color the squares of the last
move with the following command.

\commande|set_last_move_color(«color»)|\index{set_last_move_color@\lstinline+set_last_move_color+}\smallskip
\begin{description}
  \item[\meta{color}:] is a \MP{} \lstinline+color+.
\end{description}



\subsubsection{Get the number of moves}

You can get the number of half moves with the following command.

\commande|get_halfmove_number|\index{get_halfmove_number@\lstinline+get_halfmove_number+}\smallskip

This command returns a \lstinline+numeric+.

You can also get the total number of moves in the sense that they are numbered
in the \textsc{pgn} format, with the following command:

\commande|get_totalmove_number|\index{get_totalmove_number@\lstinline+get_totalmove_number+}\smallskip

This command returns a \lstinline+numeric+.

\section{Annotation}

Numerous commands allow to annotate the chessboard (arrow, color, circle, cross,
etc.).

\subsection{Arrows}

The command for drawing arrows on the chessboard is the following.

\commande|draw_arrows(«color»)(«string1»,«string2», etc.)|\index{draw_arrows@\lstinline+draw_arrows+}\smallskip
\begin{description}
\item[\meta{color}:] is a \MP{} \lstinline+color+.
\item[\meta{string1}:] is a string (between double-quotes) consisting of two
coordinates (letter and number) separated by two characters which can be 
\begin{description}
\vitem+--+ to connect the two squares in a straight line;
\vitem+-|+ to connect the two squares in a broken line, first horizontally then
vertically; 
\vitem+|-+ to connect the two squares in a broken line, first vertically then
horizontally. 
\end{description}
\end{description}

The following example illustrates the use of this command.
\begin{ExempleMP}
input mpchess;
string pgnstr;
pgnstr := "1. e4 e5 2. Nf3 Nc6 3. Nxe5 Nxe5 4. Bb5 c6";
build_chessboards_from_pgn(pgnstr);
beginfig(0);
init_backboard;
draw backboard;
show_last_move(3);
draw chessboard_step(3); % Nf3
draw_arrows(red)("f8--b4","g1|-f3");
endfig;
\end{ExempleMP}

The thickness of the arrows can be changed with the following command.
\commande|set_arrow_width(«coeff»)|\index{set_arrow_width@\lstinline+set_arrow_width+}\smallskip

\begin{description}
  \item[\meta{coeff}:] is a coefficient (\lstinline+numeric+) which allows you 
  to adjust the width of the arrows in proportion to the width of a square on the chessboard. By default, this coefficient is \lstinline+0.08+.
\end{description}

The following example illustrates the use of this command.
\begin{ExempleMP}
input mpchess;
string pgnstr;
pgnstr := "1. e4 e5 2. Nf3 Nc6 3. Nxe5 Nxe5 4. Bb5 c6";
build_chessboards_from_pgn(pgnstr);
beginfig(0);
set_black_view;
init_backboard;
draw backboard;
show_last_move(3);
draw chessboard_step(3); % Nf3
set_arrow_width(0.12);
draw_arrows(red)("f8--b4","g1|-f3");
endfig;
\end{ExempleMP}

\subsection{Coloring squares}

\mpchess also allows you to color squares with the following command.

\commande|color_square(«color»)(«coord1»,«coord2», etc.)|\index{color_square@\lstinline+color_square+}\smallskip

\begin{description}
\item[\meta{color}:] is a \MP{} \lstinline+color+.
\item[\meta{coord1}:] is a string (between double quotes) consisting of two
coordinates (letter and number). 
\end{description}

The following example illustrates the use of this command.

\begin{ExempleMP}
input mpchess
beginfig(0);
init_backboard;
draw backboard;
color_square(red)("e2","e7","c5");
init_chessboard;
draw chessboard;
endfig;
\end{ExempleMP}

This command colors the squares with a certain transparency to fit to the white
and black squares.

\subsection{Circles}

\mpchess allows you to surround squares with circles using the following command
 below. 

\commande|draw_circles(«color»)(«coord1»,«coord2», etc.)|\index{draw_circles@\lstinline+draw_circles+}\smallskip

\begin{description}
\item[\meta{color}:] is a \MP{} \lstinline+color+.
\item[\meta{coord1}:] is a string (between double quotes) consisting of two
coordinates (letter and number). 
\end{description}

The following example illustrates the use of this command.

\begin{ExempleMP}
input mpchess
beginfig(0);
init_backboard;
draw backboard;
init_chessboard;
draw chessboard;
draw_circles(green)("e2","e7","c5");
endfig;
\end{ExempleMP}



\subsection{Crosses}


\mpchess allows you to draw crosses on squares with the following command.

\commande|draw_crosses(«color»)(«coord1»,«coord2», etc.)|\index{draw_crosses@\lstinline+draw_crosses+}\smallskip

\begin{description}
\item[\meta{color}:] is a \MP{} \lstinline+color+.
\item[\meta{coord1}:] is a string (between double quotes) consisting of two
coordinates (letter and number). 
\end{description}

The following example illustrates the use of this command.

  
\begin{ExempleMP}
input mpchess
beginfig(0);
init_backboard;
draw backboard;
init_chessboard;
draw chessboard;
draw_crosses(0.7[green,black])("e2","e7","c5");
endfig;
\end{ExempleMP}

\subsection{Move comments}

\mpchess allows you to display the classic move comments of the type
``!?’’ with the following command.

\commande|draw_comment(«str»,«pos»)|\index{draw_comment@\lstinline+draw_comment+}\smallskip

\begin{description}
\item[\meta{str}:] is a string (between double-quotes) to display.
\item[\meta{pos}:] is a string (between double-quotes) consisting of a coordinate (letter and number).
\end{description}

\begin{ExempleMP}
input mpchess;
string pgnstr;
pgnstr := "1. e4 e5 2. Nf3 Nc6 3. Nxe5 Nxe5 4. Bb5 c6";
build_chessboards_from_pgn(pgnstr);
beginfig(0);
init_backboard;
draw backboard;
show_last_move(3);
draw chessboard_step(3); % Nf3
draw_comment("?!","f3");
endfig;
\end{ExempleMP}

The color of the comment annotation can be changed with the following command
following command.
\commande|set_comment_color(«color»)|\index{set_comment_color@\lstinline+set_comment_color+}\smallskip



\section{Miscellaneous}

\subsection{Reset the \lstinline+chessboard+}

To reset the internal structure storing the positions of the parts, you can use
the following command. 


\commande|clear_chessboard|\index{clear_chessboard@\lstinline+clear_chessboard+}\smallskip


\subsection{Global reset}

To reinitialize the values of the different parameters of \mpchess, you can use
the following command.

\commande|reset_mpchess|\index{reset_mpchess@\lstinline+reset_mpchess+}\smallskip


\subsection{Clip the board}

We can clip the chessboard with the following command.

\commande|clip_chessboard(«string»)|\index{clip_chessboard@\lstinline+clip_chessboard+}\smallskip


\begin{description}
\item[\meta{string}:] is a string (between double quotes)
composed of two coordinates (letter and number) separated by a dash, for example \lstinline+"a1-c6"+.
\end{description}

Here is an example of an illustration.
\begin{ExempleMP}
input mpchess;
string pgnstr;
pgnstr := "1. e4 e5 2. Nf3 Nc6 3. Nxe5 Nxe5 4. Bb5 c6";
build_chessboards_from_pgn(pgnstr);
beginfig(0);
set_black_view;
init_backboard;
draw backboard;
show_last_move(3);
draw chessboard_step(3); % Nf3
draw_comment("?!","f3");
clip_chessboard("e1-g4");
endfig;
\end{ExempleMP}

\section{To do}

Many things can be added to \mpchess. Among these, we can think of:
\begin{itemize}
\item display the captured pieces, or the differential of the captured pieces;
\item afficher le temps restant pour chaque joueur ;
\item show the accessible squares for a chosen piece (the
Lichess points);
\item display the $n$ arrows indicating the $n$ moves of the first lines of a
position (with a decreasing thickness of the arrows); 
\item adding the coordinates outside the board when it is clipped;
\item add piece themes.
\end{itemize}

\section{Complete example}

\begin{ExempleMP}[sidebyside=false]
input mpchess
string pgnstr;
pgnstr:="1. e4 e5 2. Bc4 d6 3. Nf3 Bg4 4. Nc3 g6 5. Nxe5 Bxd1";
build_chessboards_from_pgn(pgnstr);
beginfig(0);
set_white_player("Kermur de Legal");
set_black_player("Saint-Brie");
set_backboard_width(8cm);
init_backboard;
draw backboard;
show_last_move(10);
draw_comment("?","d1");
color_square(0.3[green,black])("c4","c3","e5");
color_square(0.3[red,black])("e8");
draw chessboard_step(10);
draw_arrows(0.3[green,black])("e5|-f7","c3-|d5");
draw_arrows(0.3[red,black])("c4--f7");
endfig;
\end{ExempleMP}


\printbibliography
\printindex
\end{document}



%%% Local Variables:
%%% flyspell-mode: 1
%%% ispell-local-dictionary: "fr"
%%% End:
