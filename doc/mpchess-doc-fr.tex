
% LTeX: language=fr

% mpchess: draw chess boards and positions with MetaPost
%
% Originally written by Maxime Chupin <notezik@gmail.com>,
% 2023.
%
% Distributed under the terms of the GNU free documentation licence:
%   http://www.gnu.org/licenses/fdl.html
% without any invariant section or cover text.

\documentclass[french]{ltxdoc}

\usepackage{tcolorbox}
\tcbuselibrary{listings,breakable}
%\tcbuselibrary{documentation}
\usepackage{enumitem}
\usepackage[tikz]{bclogo}
\usepackage{mflogo}
\usepackage{hologo}
\usepackage{luamplib}
\mplibtextextlabel{enable}
\usepackage{biblatex}
\addbibresource{ctan.bib}
\usepackage{wrapfig}
\usepackage{accsupp}
\usepackage{siunitx}
\usepackage{imakeidx}
%\usepackage{csquotes}
\usepackage{fancyvrb,xparse,xargs}
\usepackage[sfdefault]{FiraSans}
\usepackage[mathrm=sym]{firamath-otf}
\setmonofont{Fira Mono}
%\setmonofont{FiraCode-Regular.ttf}[BoldFont= FiraCode-Bold.ttf,ItalicFont= FiraCode-RegularItalic.otf,BoldItalicFont= FiraCode-BoldItalic.otf,Ligatures={NoCommon, NoDiscretionary, NoHistoric, NoRequired, NoContextual}]

\usepackage{xspace}
\usepackage{animate}
\newcommand{\ctan}{\textsc{ctan}}
\NewDocumentCommand{\package}{ m }{%
  \href{https://ctan.org/pkg/#1}{#1}\xspace
}

\definecolor{darkred}{rgb}{0.6,0.1,0.1}
\definecolor{vert}{rgb}{0.1,0.4,0.1}
\definecolor{bleu}{rgb}{0.2,0.2,0.6}
\definecolor{orange}{rgb}{0.6,0.4,0.}
\colorlet{code}{blue!80!black}

\usepackage[colorlinks=true,urlcolor=orange,linkcolor=orange,menucolor=black,citecolor=orange]{hyperref}

\newcommand \file       {\nolinkurl}
\renewcommand \cmd        {\texttt}
\renewcommand \code   [1] {\texorpdfstring {\texttt{\color{code}#1}} {#1}}
\renewcommand*\cs     [1] {\code{\textbackslash #1}}



\newcommand*\commande{\par\bigskip\noindent\hspace{-30pt}%
  \SaveVerb[aftersave={%
   \UseVerb{Vitem}%
  }%
  ]{Vitem}%
  }
  \newcommand\vitem[1][]{\SaveVerb[%
  aftersave={\item[\textnormal{\UseVerb[#1]{vsave}}]}]{vsave}}
\newcommand*\textme[1]{\textcolor{black}{\rmfamily\textit{#1}}}
%\renewcommand*\meta[1]{% % meta
%  \textme{\ensuremath{\langle}#1\ensuremath{\rangle}}}
\newcommand*\optstar{% % optional star
  \meta{\ensuremath{*}}\xspace}
\DefineShortVerb{\|}
\newcommand\R{\mathbf{R}}
\setlength{\fboxsep}{2pt}
\fvset{%
  codes={\catcode`\«\active \catcode`\×\active },
  defineactive={\makefancyog\makefancytimes},
  formatcom=\color{darkred},
  frame=single
}
% rendre «...» équivalent à \meta{...}
{\catcode`\«\active
  \newcommandx\makefancyog[0][addprefix=\global]{%
    \def«##1»{\meta{##1}}}}
% rendre × équivalent à \optstar
{\catcode`\×\active
  \newcommandx\makefancytimes[0][addprefix=\global]{%
    \def×{\optstar{}}}}


\newcommand\mpchess{\textbf{\textlogo{MP}}\textit{chess}\xspace}



%\addbibresource{biblio.bib}


\lstset{
  numberstyle=\footnotesize\color{vert},
  keywordstyle=\ttfamily\bfseries\color{bleu},
  basicstyle=\ttfamily,
  commentstyle=\itshape\color{vert},
  stringstyle=\ttfamily\color{orange},
  showstringspaces=false,
  language=MetaPost,
  breaklines=true,
  breakindent=30pt,
  defaultdialect=MetaPost,
  classoffset=1,% frame=tb
  morekeywords={init_backboard,set_backboard_width,set_backboard_size,set_color_theme,get_backboard_width,get_backboard_size,set_backboard_width,get_square_dim,set_white_color,set_black_color,set_coords_inside,set_coords_outside,set_coords_font,set_coords,
  set_no_coords,set_white_view,set_black_view,
  set_white_player,set_black_player,set_pieces_theme,set_players_side,init_chessboard,set_empty_chessboard,add_white_pieces,add_black_pieces,clear_areas,clear_files,clear_ranks,clear_squares,build_chessboard_from_fen,build_chessboards_from_pgn,clear_chessboard,
  color_square,draw_arrows,draw_circles,draw_comment,draw_crosses,get_halfmove_number,get_totalmove_number,
  reset_mpchess,set_black_to_move,set_last_move_color,set_comment_color,set_white_to_move,set_whos_to_move,show_last_move,unset_whos_to_move,set_arrow_width,clip_chessboard,build_chessboard_from_fen_file,build_chessboard_from_pgn_file,draw_black_main_lines,draw_black_main_lines_step,draw_white_main_lines,draw_white_main_lines_step,set_main_lines_color,set_possible_moves_color,show_possible_moves,show_possible_moves_step,hide_coordinates,hide_whos_to_move,set_black_squares_color,set_coordinates_font,set_coordinates_position,set_white_squares_color,show_coordinates,show_whos_to_move,set_board_type},
  keywordstyle=\color{darkred},
  classoffset=2,% frame=tb
  morekeywords={backboard,chessboard,chessboard_step},
  keywordstyle=\color{vert},
  classoffset=0,% frame=tb
  morekeywords={draw},
  keywordstyle=\color{bleu}
}

\makeatletter
\tcbset{%
    listing metapost/.code={%
        \def\tcbuselistingtext@input{\begin{mplibcode} input \jobname.listing; \end{mplibcode}}%
    }
}
\makeatother
\newtcblisting[auto counter,]{ExempleMP}[1][]{%
  arc=0pt,outer arc=0pt,
  colback=darkred!3,
  colframe=darkred,
  breakable,fontupper=\small,
  boxsep=0pt,left=5pt,right=5pt,top=5pt,bottom=5pt, bottomtitle =
  3pt, toptitle=3pt,
  boxrule=0pt,bottomrule=0.5pt,toprule=0.5pt, toprule at break =
  0pt, bottomrule at break = 0pt,
  listing side text,
  listing metapost,
  title=Exemple~\thetcbcounter,
  listing options={breaklines},#1
}

\newtcblisting{commandshell}{colback=black,colupper=white,colframe=black,
  arc=0pt,
  listing only,boxsep=0pt,listing
  options={style=tcblatex,language=sh},
  every listing line={\BeginAccSupp{ActualText={}}\textcolor{red}{\small\ttfamily\bfseries user \$> }}\EndAccSupp{}}


  \newtcblisting{mpcode}{
  arc=0pt,outer arc=0pt,
  colback=darkred!3,
  colframe=darkred,
  breakable,
  boxsep=0pt,left=5pt,right=5pt,top=5pt,bottom=5pt, bottomtitle =
  3pt, toptitle=3pt,
  boxrule=0pt,bottomrule=0.5pt,toprule=0.5pt, toprule at break =
  0pt, bottomrule at break = 0pt,
  listing only,boxsep=0pt,listing
  options={breaklines}
}

\newtcblisting{latexcode}{
  arc=0pt,outer arc=0pt,
  colback=darkred!3,
  colframe=darkred,
  breakable,
  boxsep=0pt,left=5pt,right=5pt,top=5pt,bottom=5pt, bottomtitle =
  3pt, toptitle=3pt,
  boxrule=0pt,bottomrule=0.5pt,toprule=0.5pt, toprule at break =
  0pt, bottomrule at break = 0pt,
  listing only,boxsep=0pt,listing
  options={breaklines,language={[LaTeX]TeX}}
}


\newcommand\pdf{\textsc{pdf}}


\newfontfamily{\chessfont}{mpchess font}
\newcommand\bP{{\chessfont \char"265F}} % black Pawn
\newcommand\bN{{\chessfont \char"265E}} % black Knight
\newcommand\bB{{\chessfont \char"265D}} % black Bishop
\newcommand\bR{{\chessfont \char"265C}} % black Rook
\newcommand\bQ{{\chessfont \char"265B}} % black Queen
\newcommand\bK{{\chessfont \char"265A}} % black King
\newcommand\wP{{\chessfont \char"2659}} % white Pawn
\newcommand\wN{{\chessfont \char"2658}} % white Knight
\newcommand\wB{{\chessfont \char"2657}} % white Bishop
\newcommand\wR{{\chessfont \char"2656}} % white Rook
\newcommand\wQ{{\chessfont \char"2655}} % white Queen
\newcommand\wK{{\chessfont \char"2654}} % white King


\usepackage[french]{babel}

\makeindex[title=Index des commandes, columns=2]




%\lstset{moredelim=*[s][\color{red}\rmfamily\itshape]{<}{>}}
%\lstset{moredelim=*[s][\color{blue}\rmfamily\itshape]{<<}{>>}}

\begin{document}

\title{{MPchess} : dessiner des plateaux d’échecs et des positions avec \hologo{METAPOST}}
\author{Maxime Chupin, \url{notezik@gmail.com}}
\date{\today}

%% === Page de garde ===================================================
\thispagestyle{empty}
\begin{tikzpicture}[remember picture, overlay]
  \node[below right, shift={(-4pt,4pt)}] at (current page.north west) {%
    \includegraphics{fond.pdf}%
  };
\end{tikzpicture}%


\noindent
{\Huge \mpchess}\par\medskip
\noindent
{\Large  dessiner des plateaux d’échecs et des positions avec \hologo{METAPOST}}\\[1cm]
\parbox{0.6\textwidth}{
  \begin{mplibcode}
    input mpchess
    string pgnstr;
    pgnstr:="1. e4 e5 2. Bc4 d6 3. Nf3 Bg4 4. Nc3 g6 5. Nxe5 Bxd1 ";
    build_chessboards_from_pgn(pgnstr);
    beginfig(0);
    set_backboard_width(8cm);
    init_backboard;
    draw backboard;
    show_last_move(10);
    draw_comment("?","d1");
    color_square(0.3[green,black])("c4","c3","e5");
    color_square(0.3[red,black])("e8");
    draw chessboard_step(10);
    draw_arrows(0.3[green,black])("e5|-f7","c3-|d5");
    draw_arrows(0.3[red,black])("c4--f7");
    endfig;
  \end{mplibcode}
}\hfill
\parbox{0.5\textwidth}{\Large\raggedleft
  \textbf{Contributeur}\\
  Maxime \textsc{Chupin}\\
  \url{notezik@gmail.com}
}
\vfill
\begin{center}
  Version 0.7, 20 juillet 2023 \\
  \url{https://plmlab.math.cnrs.fr/mchupin/mpchess}
\end{center}
%% == Page de garde ====================================================
\newpage

%\maketitle

\begin{abstract}
Ce package \MP{} permet de dessiner des plateaux d’échecs et des positions.
L’apparence des dessins se veut moderne et largement inspiré de ce que propose
l’excellent site web \url{Lichess.org}.
S’appuyer sur \MP{} permet sans doute plus de flexibilité graphique que les
excellent packages \LaTeX{}.
\end{abstract}


\begin{center}
  \url{https://plmlab.math.cnrs.fr/mchupin/mpchess}
  \url{https://github.com/chupinmaxime/mpchess}
\end{center}

\tableofcontents

\bigskip

\begin{tcolorbox}[ arc=0pt,outer arc=0pt,
  colback=darkred!3,
  colframe=darkred,
  breakable,
  boxsep=0pt,left=5pt,right=5pt,top=5pt,bottom=5pt, bottomtitle =
  3pt, toptitle=3pt,
  boxrule=0pt,bottomrule=0.5pt,toprule=0.5pt, toprule at break =
  0pt, bottomrule at break = 0pt,]
  \itshape
  Ce package est en version beta, n’hésitez pas à faire remonter les bugs, ainsi
  que les demandes d’amélioration. 
\end{tcolorbox}

\section{Installation}

\mpchess est sur le \ctan{} et peut être installé via le gestionnaire de package
de votre distribution.

\begin{center}
  \url{https://www.ctan.org/pkg/mpchess}
\end{center}


\subsection{Avec la \TeX live sous Linux ou macOS}

Pour installer \mpchess avec \TeX live, il vous faudra créer le répertoire
\lstinline+texmf+ dans votre \lstinline+home+.

\begin{commandshell}
mkdir ~/texmf
\end{commandshell}

Ensuite, il faudra y placer les fichiers \lstinline+.mp+ dans le répertoire \begin{center}
  \lstinline+~/texmf/metapost/mpchess/+
\end{center}

\mpchess est constitué de 7 fichiers \hologo{METAPOST} :
\begin{itemize}
  \item \verb+mpchess.mp+;
  \item \verb+mpchess-chessboard.mp+;
  \item \verb+mpchess-pgn.mp+;
  \item \verb+mpchess-fen.mp+;
  \item \verb+mpchess-cburnett.mp+;
  \item \verb+mpchess-pieces.mp+;
  \item \verb+mpchess-skak.mp+.
\end{itemize}

Une fois fait cela, \mpchess sera chargé avec le classique
\begin{mpcode}
input mpchess
\end{mpcode}

\subsection{Avec Mik\TeX{} et Windows}

Ces deux systèmes sont inconnus de l’auteur de \mpchess, ainsi, nous renvoyons à
leurs documentations pour y ajouter des packages locaux :
\begin{center}
  \url{http://docs.miktex.org/manual/localadditions.html}
\end{center}



\subsection{Dépendances}

\mpchess dépend des packages \MP: \package{hatching} et, si
\mpchess n’est pas utilisé avec \hologo{LuaLaTeX} et \package{luamplib},
\package{latexmp}.
\section{Pourquoi ce package et philosophie générale}

Il existe déjà des packages \LaTeX{} pour dessiner des plateaux d’échecs et des
positions dont le très bon \package{xskak}~\cite{ctan-xskak} couplé avec le
package \package{chessboard}~\cite{ctan-chessboard}. Ulrike Fisher a réalisé là un
travail d’amélioration, de maintient, et nous a fourni d’excellents outils pour
réaliser des diagrammes d’échecs et de traiter les différents formats de
descriptions de parties\footnote{Elle a même développé le
package~\package{chessfss} pour gérer diverses fontes d’échec.}. Les
documentations de ces packages sont de très bonnes qualités.

Plusieurs choses ont motivé la création de \mpchess. Tout d’abord, avec
\package{chessboard} l’ajout d’ensemble de pièces n’est pas très aisé, car cela
repose sur des fontes. De plus, je trouve que le dessin de diagrammes de parties
d’échec est quelque chose de très graphique, et que le passage par un langage
dédié au dessin offre plus de souplesse et quoi de mieux que \MP~\cite{ctan-metapost}.


Avec \mpchess, on construit l’image finale du plateau d’échec avec les pièces
par couches successives. Ainsi, on commencera par produire et dessiner le
plateau (\lstinline+backboard+), que l’on pourra modifier en colorant par
exemple certaines cases, ensuite on ajoutera les pièces de la position
(\lstinline+chessboard+), et enfin, on pourra annoter le tout avec des marques, des couleurs, des flèches, etc.

Par ailleurs, \mpchess produit des images proches graphiquement de ce que peut
fournir l’excellent site \emph{open source} \url{https://lichess.org}. Vous
verrez que les couleurs, les pièces et l’aspect général sont largement inspirés
de ce que propose ce site.

\section{Plateau}

Le plateau est appelé avec \mpchess{} \lstinline+backboard+.
Il faudra initialiser le plateau avant de le dessiner. Cela se fait avec la
commande suivante :\par

\commande|init_backboard|\smallskip\index{init_backboard@\lstinline+init_backboard+}


Cette commande construit une \lstinline+picture+ de \MP{} nommée
\mbox{\lstinline+backboard+.} Il faudra ensuite la tracer comme l’illustre l’exemple
suivant.

\begin{ExempleMP}
input mpchess

beginfig(0);
init_backboard;
draw backboard;
endfig;
\end{ExempleMP}

Cette initialisation permettra de prendre en compte les différentes options et
fonctionnalités que nous allons décrire dans la suite.
\subsection{Réglage des tailles}
Lors de la création du \lstinline+backboard+, on peut décider de la largeur de celui-ci. Cela se fait grâce à la commande suivante :\par\bigskip

\commande|set_backboard_width(«dim»)|\smallskip\index{set_backboard_width@\lstinline+set_backboard_width+}

\begin{description}
  \item[\meta{dim}:] est la largeur de plateau de jeu souhaitée (avec l’unité). Par défaut, cette dimension est à \SI{5}{cm}.
\end{description}

L’utilisation de cette commande est illustré à l’exemple~\ref{ex:widthcase}. Cette commande est à utilisée avant \lstinline+init_backboard+ pour qu’elle soit prise en compte à la création de l’image.
\bigskip

On peut récupérer la dimension du plateau de jeu par la commande suivante.

\commande|get_backboard_width|\smallskip\index{get_backboard_width@\lstinline+get_backboard_width+}

Cette commande retourne un type \lstinline+numeric+.

\subsection{Nombre de case}
Par défaut, le plateau de jeu contient 64 cases ($8\times 8$). On peut modifier
cela avec la commande suivante:

\commande|set_backboard_size(«nbr»)|
\smallskip\index{set_backboard_size@\lstinline+set_backboard_size+}

\begin{description}
  \item[\meta{nbr}:] est le nombre de cases souhaité. Le plateau sera alors carré de taille \meta{nbr}$\times$\meta{nbr}. Par défaut ce nombre est à 8.
\end{description}

Encore une fois, cette commande est à utilisée avant \lstinline+init_backboard+ pour qu’elle soit prise en compte comme le montre l’exemple suivant.

\begin{ExempleMP}[label=ex:widthcase]
input mpchess

beginfig(0);
set_backboard_width(3cm);
set_backboard_size(6);
init_backboard;
draw backboard;
endfig;
\end{ExempleMP}

Pour obtenir la taille du plateau de jeu, on pourra utiliser la commande
suivante. 

\commande|get_backboard_size|\smallskip\index{get_backboard_size@\lstinline+get_backboard_size+}

Cette commande retourne un type \lstinline+numeric+.

\subsection{Dimension d’une case}

En fonction du nombre de cases sur le plateau et la largeur prescrite pour le
plateau, \mpchess calcule la dimension (largeur ou hauteur) d’une case. Cela
sert d’unité générale. Pour l’obtenir, on utilisera la commande suivante.

\commande|get_square_dim|\smallskip\index{get_square_dim@\lstinline+get_square_dim+}

Cette commande retourne un \lstinline+numeric+.

\subsection{Réglage du thème de couleur}

\subsubsection{Thèmes prédéfinis}

Plusieurs thèmes de couleurs sont accessibles. Pour choisir un thème de couleur, on utilisera la commande suivante :

\commande|set_color_theme(«string»)|\smallskip\index{set_color_theme@\lstinline+set_color_theme+}

\begin{description}
  \item[\meta{string}] peut valoir :
\begin{itemize}
\item \lstinline+"BlueLichess"+ (thème par défaut);
\item \lstinline+"BrownLichess"+ ;
\item \lstinline+"GreenLichess"+;
\item \lstinline+"PinkPyramidalLichess"+;
\item \lstinline+"Wood"+;
\item ou \lstinline+"Classical"+.
\end{itemize}
\end{description}

Les exemples suivants montrent les résultats obtenus.
\begin{ExempleMP}
input mpchess
beginfig(0);
set_color_theme("BrownLichess");
init_backboard;
draw backboard;
endfig;
\end{ExempleMP}

Le tableau~\ref{tab:color} montre les résultats des différents thèmes fournis
par \mpchess. 

\begin{table}
  \centering
\begin{tabular}{cc}

Thème \lstinline+"GreenLichess"+&Thème \lstinline+"Classical"+\\
\begin{mplibcode}
  input mpchess
  beginfig(0);
  init_chessboard;
  set_backboard_width(4cm);
  set_color_theme("GreenLichess");
  init_backboard;
  draw backboard;
  endfig;
\end{mplibcode}
&\begin{mplibcode}
  input mpchess
  beginfig(0);
  init_chessboard;
  set_backboard_width(4cm);
  set_color_theme("Classical");
  init_backboard;
  draw backboard;
  endfig;
\end{mplibcode}\\
Thème \lstinline+"PinkPyramidalLichess"+&Thème  \lstinline+"Wood"+\\
\begin{mplibcode}
  input mpchess
  beginfig(0);
  init_chessboard;
  set_backboard_width(4cm);
  set_color_theme("PinkPyramidalLichess");
  init_backboard;
  draw backboard;
  endfig;
\end{mplibcode}&
\begin{mplibcode}
  input mpchess
  beginfig(0);
  init_chessboard;
  set_backboard_width(4cm);
  set_color_theme("Wood");
  init_backboard;
  draw backboard;
  endfig;
\end{mplibcode}\\
\end{tabular}
\caption{Les différents thèmes de couleur fournis par \mpchess.}\label{tab:color}
\end{table}

\subsubsection{Configuration d’un thème personnel}

\paragraph{Couleurs.}
Un thème de couleur conciste en la définition de deux couleurs.
Celles-ci peuvent se définir avec les commandes suivantes \footnote{Attention,
lors du passage à la version 0.6, \lstinline+set_white_color+ est devenu
\lstinline+set_white_squares_color+ et \lstinline+set_black_color+ est devenu
\lstinline+set_black_squares_color+.}.

\commande|set_white_squares_color(«color»)|\index{set_white_squares_color@\lstinline+set_white_squares_color+}\par
\commande|set_black_squares_color(«color»)|\index{set_black_squares_color@\lstinline+set_black_squares_color+}\smallskip

\meta{color} est une \lstinline+color+ \MP.

\begin{ExempleMP}
input mpchess
beginfig(0);
set_white_squares_color((0.9,0.8,0.8));
set_black_squares_color((0.7,0.6,0.6));
init_backboard;
draw backboard;
endfig;
\end{ExempleMP}

\paragraph{Type de coloriage.}


On peut choisir un type de coloriage avec la commande suivante:

\commande|set_board_type(«string»)|\index{set_board_type@\lstinline+set_board_type+}\smallskip

Les trois types de coloriage fournis par \mpchess{} se choisissent avec 
\begin{description}
  \item[\meta{string}] qui peut valoir:
\begin{itemize}
\item \lstinline+"flat"+,  coloriage à plat (défaut);
\item \lstinline+"pyramidal"+, coloriage \emph{pyramidal} à la Lichess; 
\item \lstinline+"wood"+, imitation bois.
\end{itemize}
\end{description}

Voici un exemple de paramétrage de couleur et de type de coloriage. 

\begin{ExempleMP}
input mpchess
beginfig(0);
set_white_squares_color((0.9,0.8,0.8));
set_black_squares_color((0.7,0.6,0.6));
set_board_type("wood");
init_backboard;
draw backboard;
endfig;
\end{ExempleMP}

\subsection{Affichage des coordonnées}

Vous avez pu constater dans les divers exemples que par défaut, les coordonnées
sont, comme le fait le site Lichess, inscrites en petit à l’intérieur des
cases.

\mpchess permet de choisir le positionnement de ces coordonnées à l’extérieur ou
à l’intérieur du plateau avec la commande suivante\footnote{Attention,
en version 0.6, \lstinline+set_coords_inside+ et \lstinline+set_coords_outside+
ont été remplacés par 
\lstinline+set_coordinates_position+.}.

\commande|set_coordinates_position(«string»)|\index{set_coordinates_position@\lstinline+set_coordinates_position+}\smallskip

\begin{description}
  \item[\meta{string}] peut valoir :
\begin{itemize}
\item \lstinline+"inside"+ (par défaut);
\item \lstinline+"outside"+.
\end{itemize}
\end{description}


Le résultat est alors le suivant.

\begin{ExempleMP}
input mpchess
beginfig(0);
set_coordinates_position("outside");
init_backboard;
draw backboard;
endfig;
\end{ExempleMP}

Vous pouvez constater dans cette documentation qu’avec \package{luamplib} et
\LaTeX, la fonte est la fonte du document courant. Pour tracer ces lettres et
ces chiffres, \mpchess utilise l’opérateur \MP{} \lstinline+infont+ et la fonte
est réglée à \lstinline+defaultfont+ par défaut\footnote{Avec \package{luamplib}
l’opérateur \lstinline+infont+ est redéfini et son argument est simplement
ignoré, ainsi, il n’est pas possible de modifier la fonte de composition des
coordonnées.}. On peut modifier cette fonte avec la commande suivante\footnote{Attention,
en version 0.6, \lstinline+set_coords_font+ est devenue 
\lstinline+set_coordinates_font+.}.

\commande|set_coordinates_font(«font»)|\index{set_coordinates_font@\lstinline+set_coordinates_font+}\smallskip

Il faudra alors utiliser les conventions de nommage propres à l’opérateur
\lstinline+infont+ de \MP{} et nous renvoyons à la
documentation~\cite{ctan-metapost} pour plus de détails.

On pourra aussi supprimer les coordonnées avec la commande suivante\footnote{Attention,
dans la version 0.6, \lstinline+set_no_coords+ est devenu
\lstinline+hide_coordinates+ et \lstinline+set_coords+ est devenu
\lstinline+show_coordinates+}.

\commande|hide_coordinates|\index{hide_coordinates@\lstinline+hide_coordinates+}\smallskip

Et la commande inverse aussi existe.

\commande|show_coordinates|\index{show_coordinates@\lstinline+show_coordinates+}\smallskip


\subsection{Vue blanche ou noire}

Pour choisir si l’on souhaite voir le plateau du côté blanc ou noir, \mpchess fournit deux commandes.

\commande|set_white_view|\index{set_white_view@\lstinline+set_white_view+}\smallskip

\commande|set_black_view|\index{set_black_view@\lstinline+set_black_view+}\smallskip

Par défaut, on voit l’échiquier côté blanc.

\subsection{Noms des joueurs}

On peut renseigner les noms des joueuses ou des joueurs pour qu’ils soient notés
autour de l’échiquier. Ceci se fait avec les commandes suivantes.
\commande|set_white_player(«string»)|\index{set_white_player@\lstinline+set_white_player+}\smallskip

\commande|set_black_player(«string»)|\index{set_black_player@\lstinline+set_black_player+}\smallskip

\begin{description}
  \item[\meta{string} :] est la chaîne de caractères interprétée par \LaTeX{} à afficher.
\end{description}

\begin{ExempleMP}
input mpchess
beginfig(0);
set_white_player("\textbf{GM} Kasparov");
set_black_player("\textbf{GM} Kramnik");

init_backboard;
draw backboard;
endfig;
\end{ExempleMP}

Il est possible de placer les noms sur le côté droit du plateau sans les
bandeaux noirs présents par défaut. Cela se produit soit si les coordonnées sont
imprimées à l’extérieur du plateau, soit si la commande suivante est utilisée.

\commande|set_players_side|\index{set_players_side@\lstinline+set_players_side+}\smallskip

\begin{ExempleMP}[righthand width=0.6\linewidth]
input mpchess
beginfig(0);
set_white_player("\textbf{GM} Kasparov");
set_black_player("\textbf{GM} Kramnik");
set_players_side;
init_backboard;
draw backboard;
endfig;
\end{ExempleMP}



\section{Pièces et positions}

\mpchess, comme décrit plus haut, construit le graphique d’une position d’échec
couche par couche. Cette partie est dédiée à la configuration des pièces et des
positions.

En interne, \mpchess construit un tableau sur la grille du plateau. Ensuite, des
macros permettent de générer une \lstinline+picture+ à dessiner \emph{par
  dessus} le plateau (\lstinline+backboard+).

\subsection{Réglage du thème des pièces}

\mpchess fournit pour l’instant trois thèmes de pièces. Le thème par défaut est
appelé \lstinline+mpchess+. Il a été designé pour ce package \hologo{METAPOST}.
Il a été proposé au projet Lichess, et a été accepté. Ainsi, vous aurez aussi
accès à l’ensemble de pièces \lstinline+mpchess+ avec Lichess\footnote{Les
projets libres se nourrissent mutuellement ! Même si évidemment, ce package a
bien plus emprunté à Lichess que le contraire.} ! Un autre thème est emprunté à
Lichess (\lstinline+cburnett+) et  l’autre est emprunté au package~\package{skak}~\cite{ctan-skak}\footnote{Qui
  fournit le code \MF{} pour la fonte de pièces d’échec, code qui a été
  facilement adapté en \MP{} pour \mpchess.}.

Pour choisir le thème on utilisera la commande suivante.

\commande|set_pieces_theme(«string»)|\index{set_pieces_theme@\lstinline+set_pieces_theme+}\smallskip

\begin{description}
  \item[\meta{string}:] peut valoir:
\begin{itemize}
  \item \lstinline+"mpchess"+ (valeur par défaut), pour obtenir l’ensemble de pièces spécialement
  conçu pour ce package ;
  \item \lstinline+"cburnett"+, pour obtenir l’ensemble de pièces nommé \emph{cburnett} de Lichess;
\item \lstinline+"skak"+, pour obtenir l’ensemble de pièces du package~\package{skak}.
\end{itemize}
\end{description}

Le tableau~\ref{tab:pieces} montre le résultat des trois ensembles de pièces.
\begin{table}
  \centering
\begin{tabular}{cc}

Thème \lstinline+cburnett+&Thème \lstinline+mpchess+\\
\begin{mplibcode}
  input mpchess
  beginfig(0);
  init_chessboard;
  set_backboard_width(4cm);
  set_pieces_theme("cburnett");
  hide_whos_to_move;
  init_backboard;
  draw backboard;
  draw chessboard_step(0);
  endfig;
\end{mplibcode}
&\begin{mplibcode}
  input mpchess
  beginfig(0);
  init_chessboard;
  set_backboard_width(4cm);
  set_pieces_theme("mpchess");
  hide_whos_to_move;
  init_backboard;
  draw backboard;
  draw chessboard_step(0);
  endfig;
\end{mplibcode}\\
Thème \lstinline+skak+&\\
\begin{mplibcode}
  input mpchess
  beginfig(0);
  init_chessboard;
  set_backboard_width(4cm);
  set_pieces_theme("skak");
  hide_whos_to_move;
  init_backboard;
  draw backboard;
  draw chessboard_step(0);
  endfig;
\end{mplibcode}&\\
\end{tabular}
\caption{Les différents thèmes de pièces fournits par \mpchess.}\label{tab:pieces}
\end{table}



\subsection{Trait}

\mpchess indique qui a le trait entre les blancs et les noirs. Ceci ce fait par
un petit triangle coloré (blanc ou noir) à l’extérieur du plateau (que vous
pourrez observer dans les nombreux exemples suivants).

Pour spécifier qui a le trait on utilisera les commandes suivantes.

\commande|set_white_to_move|\index{set_white_to_move@\lstinline+set_white_to_move+}\smallskip


\commande|set_black_to_move|\index{set_black_to_move@\lstinline+set_black_to_move+}\smallskip

Par défaut, c’est aux blancs de jouer, et cette information est affichée.

Pour activer ou désactiver l’affichage du trait, on utilisera une des deux commandes suivantes\footnote{Attention,
lors du passage à la version 0.6, \lstinline+set_whos_to_move+ et
\lstinline+unset_whos_to_move+ sont devenus \lstinline+show_whos_to_move+ et
\lstinline+hide_whos_to_move+.}.

\commande|show_whos_to_move|\index{show_whos_to_move@\lstinline+show_whos_to_move+}\smallskip


\commande|hide_whos_to_move|\index{hide_whos_to_move@\lstinline+hide_whos_to_move+}\smallskip

\subsection{Dessiner une position}

Les commandes décrites ci-dessous permet de construire une position de plusieurs
façons (ajout de pièces une à une, lecture de fichier \textsc{fen}, etc.). Une
fois une position construite, on peut la tracer grâce à la commande suivante qui
génère une \lstinline+picture+.

\commande|chessboard|\index{chessboard@\lstinline+chessboard+}\smallskip \label{com:chessboard}

L’utilisation de cette commande va être largement illustrée dans les exemples suivants.

\subsection{Construire une position}

\subsubsection{Initialisation}

Pour obtenir la position initiale d’une partie, il suffit d’utiliser la commande suivante.

\commande|init_chessboard|\index{init_chessboard@\lstinline+init_chessboard+}\smallskip

\begin{ExempleMP}
input mpchess
beginfig(0);
init_backboard;
draw backboard;
init_chessboard;
draw chessboard;
endfig;
\end{ExempleMP}

On pourra aussi initialiser un \lstinline+chessboard+ vide grâce à la commande suivante.

\commande|set_empty_chessboard|\index{set_empty_chessboard@\lstinline+set_empty_chessboard+}\smallskip

\subsubsection{Ajout de pièces}

On peut ajouter des pièces pour construire une position grâce aux deux commandes suivantes.

\commande|add_white_pieces(«piece1»,«piece2»,etc.)|\index{add_white_pieces@\lstinline+add_white_pieces+}\smallskip


\commande|add_black_pieces(«piece1»,«piece2»,etc.)|\index{add_black_pieces@\lstinline+add_black_pieces+}\smallskip

Ces commandes prennent des listes de \textbf{\meta{piece}} qui sont des chaînes
de caractères qui décrivent la pièce et la position en utilisant la notation
algébrique. Il n’y a pas de limitation au nombre de pièces dans la liste.

L’exemple suivant illustre l’utilisation de ces commandes.
\begin{ExempleMP}
input mpchess
beginfig(0);
init_backboard;
draw backboard;
set_empty_chessboard;
add_white_pieces("e1","Kd2");
add_black_pieces("e7","f6","Kb8");
draw chessboard;
endfig;
\end{ExempleMP}


\subsubsection{Suppression de pièces}

\mpchess fournit plusieurs commandes permettant de supprimer des éléments d’une position.

La première commande permet de supprimer un élément d’une case. Cette commande permet de prendre une liste de cases, en utilisant la notation algébrique.

\commande|clear_squares(«square1»,«square2»,etc.)|\index{clear_squares@\lstinline+clear_squares+}\smallskip

Les \textbf{\meta{square1}}, \textbf{\meta{square2}}, etc., sont des chaînes de caractères, par exemple \lstinline+"a3"+.
\medskip

La commande suivante permet de supprimer un ensemble de cases dans une région
déterminé par deux coordonnées sur le plateau. Cette commande permet de prendre
une liste de régions.

\commande|clear_areas(«area1»,«area2»,etc.)|\index{clear_areas@\lstinline+clear_areas+}\smallskip

Les \textbf{\meta{area1}}, \textbf{\meta{area2}}, etc., sont des chaînes de caractères constituées de deux coordonnées séparées par un tiret, par exemple \lstinline+"a3-g7"+.
\medskip

La commande suivante permet de supprimer l’ensemble des cases d’une colonne
déterminé par une lettre sur le plateau. Cette commande permet de prendre
une liste de colonnes.

\commande|clear_files(«file1»,«file2»,etc.)|\index{clear_files@\lstinline+clear_files+}\smallskip

Les \textbf{\meta{file1}}, \textbf{\meta{file2}}, etc., sont des chaînes de caractères constituées d’une lettre, par exemple \lstinline+"a"+.
\medskip

La commande suivante permet de supprimer l’ensemble des cases d’une ligne
déterminé par un nombre sur le plateau. Cette commande permet de prendre
une liste de lignes.

\commande|clear_ranks(«rank1»,«rank2»,etc.)|\index{clear_ranks@\lstinline+clear_ranks+}\smallskip

Les \textbf{\meta{rank1}}, \textbf{\meta{rank2}}, etc., sont des chaînes de caractères constituées d’un nombre, par exemple \lstinline+"4"+.

L’utilisation de l’ensemble des ces commandes est illustrée dans l’exemple suivant.

\begin{ExempleMP}
input mpchess
beginfig(0);
init_backboard;
draw backboard;
init_chessboard;
clear_squares("a1","b2");
clear_areas("c2-d7");
clear_files("f","h");
clear_ranks("8");
draw chessboard;
endfig;
\end{ExempleMP}



\subsection{Lecture de données au format \textsc{fen}}

\mpchess permet de lire une position au format \textsc{fen} grâce à la commande suivante.
\commande|build_chessboard_from_fen(«string»)|\index{build_chessboard_from_fen@\lstinline+build_chessboard_from_fen+}\smallskip

\begin{description}
\item[\meta{string}:] est une chaîne de caractères décrivant une position au
format \textsc{fen}. Notons que toutes les informations après l’information du
\emph{trait} (\texttt{w} ou \texttt{b}) sont ignorées.
\end{description}

\begin{ExempleMP}
input mpchess;
beginfig(0);
init_backboard;
draw backboard;
string fenstr;
fenstr := "7r/2p1kp1p/p1B2p2/1pb5/8/2PP4/PP1N1PPP/R5K1 b - - 2 19";
build_chessboard_from_fen(fenstr);
draw chessboard;
endfig;
\end{ExempleMP}

Il est aussi possible de lire un fichier externe contenant sur la première ligne
une chaîne de caractères au format \textsc{fen} avec la commande suivante. 

\commande|build_chessboard_from_fen_file(«string»)|\index{build_chessboard_from_fen_file@\lstinline+build_chessboard_from_fen_file+}\smallskip

\begin{description}
\item[\meta{string}:] est une chaîne de caractères (entre double-quotes)
indiquant le nom du fichier à lire. 
\end{description}

\begin{ExempleMP}
input mpchess;
beginfig(0);
init_backboard;
draw backboard;
build_chessboard_from_fen_file("test.fen");
draw chessboard;
endfig;
\end{ExempleMP}


  
\subsection{Lecture de données au format \textsc{pgn}}

\mpchess permet aussi de lire une chaîne de caractères au format \textsc{pgn}.
Attention, il s’agit d’une gestion partielle du format, en particulier
\mpchess ne gère pas les \emph{tags} du format. En réalité, \mpchess ne traite
que la chaîne de caractères décrivant les coups joués. De même, le format
\textsc{pgn} accepté par \mpchess n’accepte ni les variantes ni les
commentaires.  

Lorsqu’une telle fonctionnalité est utilisé, \mpchess stocke toutes les
positions intermédiaires et permet ainsi de les représenter.

Pour construire les positions, on utilisera la commande suivante.

\commande|build_chessboards_from_pgn(«string»)|\index{build_chessboards_from_pgn@\lstinline+build_chessboards_from_pgn+}\smallskip



Une fois le positions construites, on pourra les représenter grâce à la commande suivante.

\commande|chessboard_step(«int»)|\index{chessboard_step@\lstinline+chessboard_step+}\smallskip
\begin{description}
\item[\meta{int}:] est le numéro du l’étape. La configuration initiale est numérotée 0, et ensuite, chaque coup, blanc ou noir, est numéroté.
\end{description}

Cette commande, comme \lstinline+chessboard+ (voir page~\pageref{com:chessboard}), retourne une \lstinline+picture+.

L’exemple suivant illustre l’utilisation de ces commandes.

\begin{ExempleMP}
input mpchess;
string pgnstr;
pgnstr := "1. e4 e5 2. Nf3 Nc6 3. Nxe5 Nxe5 4. Bb5 c6";
build_chessboards_from_pgn(pgnstr);
beginfig(0);
init_backboard;
draw backboard;
draw chessboard_step(3); % Nf3
endfig;
\end{ExempleMP}

Il est aussi possible de lire un fichier externe contenant sur la première ligne
une chaîne de caractères au format \textsc{pgn} avec la commande suivante. 

\commande|build_chessboard_from_pgn_file(«string»)|\index{build_chessboard_from_pgn_file@\lstinline+build_chessboard_from_pgn_file+}\smallskip

\begin{description}
\item[\meta{string}:] est une chaîne de caractères (entre double-quotes)
indiquant le nom du fichier à lire. 
\end{description}

\begin{ExempleMP}
input mpchess;
build_chessboards_from_pgn_file("test.pgn");
beginfig(0);
init_backboard;
draw backboard;
draw chessboard_step(4); % Nc6
endfig;
\end{ExempleMP}



\subsubsection{Montrer le dernier coup}

On peut afficher automatiquement le dernier coup grâce à la commande suivante.

\commande|show_last_move(«int»)|\index{show_last_move@\lstinline+show_last_move+}\smallskip

\begin{description}
\item[\meta{int}:] est le numéro du l’étape. La configuration initiale est numérotée 0, et ensuite, chaque coup, blanc ou noir, est numéroté.
\end{description}

Cette commande colorie en transparence les deux cases de départ et d’arrivée du dernier coup. Ainsi, elle doit être utilisée entre le dessin du plateau (\lstinline+draw backboard+) et le dessin des pièces (\lstinline+draw chessboard_step(i)+).

\begin{ExempleMP}
input mpchess;
string pgnstr;
pgnstr := "1. e4 e5 2. Nf3 Nc6 3. Nxe5 Nxe5 4. Bb5 c6";
build_chessboards_from_pgn(pgnstr);
beginfig(0);
init_backboard;
draw backboard;
show_last_move(3);
draw chessboard_step(3); % Nf3
endfig;
\end{ExempleMP}

On pourra configurer la couleur utilisée pour colorier en transparence les cases du dernier coup grâce à la commande suivante.

\commande|set_last_move_color(«color»)|\index{set_last_move_color@\lstinline+set_last_move_color+}\smallskip
\begin{description}
  \item[\meta{color}:] est une \lstinline+color+ \MP.
\end{description}



\subsubsection{Obtenir le nombre de coups}

On pourra récupérer le nombre de \emph{demi}-coups grâce à la
commande suivante.

\commande|get_halfmove_number|\index{get_halfmove_number@\lstinline+get_halfmove_number+}\smallskip

Cette commande retourne un \lstinline+numeric+.

On pourra aussi récupérer le nombre de coups \og{}total\fg{} au sens où il sont numéroté dans le format \textsc{pgn}, grâce à la commande suivante :

\commande|get_totalmove_number|\index{get_totalmove_number@\lstinline+get_totalmove_number+}\smallskip

Cette commande retourne un \lstinline+numeric+.

\section{Annotation}

De nombreuses commandes permettent d’annoter l’échiquier (flèche, couleur, cercle, croix, etc.).

\subsection{Flèches}

La commande pour tracer des flèches sur l’échiquier est la suivante.

\commande|draw_arrows(«color»)(«string1»,«string2», etc.)|\index{draw_arrows@\lstinline+draw_arrows+}\smallskip
\begin{description}
\item[\meta{color}:] est une \lstinline+color+ \MP.
\item[\meta{string1}:] est une chaîne de caractères (entre double-quotes) constituée de deux coordonnées (lettre et chiffre) séparés par deux caractères qui peuvent être
\begin{description}
\vitem+--+ pour relier les deux cases en ligne droite ;
\vitem+-|+ pour relier les deux cases en ligne brisée, d’abord horizontalement puis verticalement ;
\vitem+|-+ pour relier les deux cases en ligne brisée, d’abord verticalement puis horizontalement.
\end{description}
\end{description}

L’exemple suivant illustre l’utilisation de cette commande.
\begin{ExempleMP}
input mpchess;
string pgnstr;
pgnstr := "1. e4 e5 2. Nf3 Nc6 3. Nxe5 Nxe5 4. Bb5 c6";
build_chessboards_from_pgn(pgnstr);
beginfig(0);
init_backboard;
draw backboard;
show_last_move(3);
draw chessboard_step(3); % Nf3
draw_arrows(red)("f8--b4","g1|-f3");
endfig;
\end{ExempleMP}

On pourra modifier l’épaisseur des flèches grâce à la commande suivante.
\commande|set_arrow_width(«coeff»)|\index{set_arrow_width@\lstinline+set_arrow_width+}\smallskip

\begin{description}
  \item[\meta{coeff}:] est un coefficient (\lstinline+numeric+) qui permet de
  régler la largeur des flèches proportionnellement à la largeur d’une case de
  l’échiquier. Par défaut, ce coefficient vaut \lstinline+0.08+.
\end{description}

L’exemple suivant illustre l’utilisation de cette commande.
\begin{ExempleMP}
input mpchess;
string pgnstr;
pgnstr := "1. e4 e5 2. Nf3 Nc6 3. Nxe5 Nxe5 4. Bb5 c6";
build_chessboards_from_pgn(pgnstr);
beginfig(0);
set_black_view;
init_backboard;
draw backboard;
show_last_move(3);
draw chessboard_step(3); % Nf3
set_arrow_width(0.12);
draw_arrows(red)("f8--b4","g1|-f3");
endfig;
\end{ExempleMP}


\subsection{Coloration de cases}
\mpchess permet aussi de colorer des cases grâce à la commande suivante.

\commande|color_square(«color»)(«coord1»,«coord2», etc.)|\index{color_square@\lstinline+color_square+}\smallskip

\begin{description}
\item[\meta{color}:] est une \lstinline+color+ \MP.
\item[\meta{coord1}:] est une chaîne de caractères (entre double-quotes) constituée de deux coordonnées (lettre et chiffre).
\end{description}

L’exemple suivant permet d’illustrer l’utilisation de cette commande.

\begin{ExempleMP}
input mpchess
beginfig(0);
init_backboard;
draw backboard;
color_square(red)("e2","e7","c5");
init_chessboard;
draw chessboard;
endfig;
\end{ExempleMP}

Cette commande colorie les cases avec une certaine transparence pour s’adapter
aux cases blanches et noires.

\subsection{Cercles}
\mpchess permet d’entourer des cases avec des cercles grâce à la commande
suivante.
\commande|draw_circles(«color»)(«coord1»,«coord2», etc.)|\index{draw_circles@\lstinline+draw_circles+}\smallskip

\begin{description}
\item[\meta{color}:] est une \lstinline+color+ \MP.
\item[\meta{coord1}:] est une chaîne de caractères (entre double-quotes) constituée de deux coordonnées (lettre et chiffre).
\end{description}

L’exemple suivant permet d’illustrer l’utilisation de cette commande.

\begin{ExempleMP}
input mpchess
beginfig(0);
init_backboard;
draw backboard;
init_chessboard;
draw chessboard;
draw_circles(green)("e2","e7","c5");
endfig;
\end{ExempleMP}



\subsection{Croix}


\mpchess permet de tracer des croix sur des cases  grâce à la commande
suivante.
\commande|draw_crosses(«color»)(«coord1»,«coord2», etc.)|\index{draw_crosses@\lstinline+draw_crosses+}\smallskip

\begin{description}
  \item[\meta{color}:] est une \lstinline+color+ \MP.
  \item[\meta{coord1}:] est une chaîne de caractères (entre double-quotes) constituée de deux coordonnées (lettre et chiffre).
  \end{description}

L’exemple suivant permet d’illustrer l’utilisation de cette commande.

\begin{ExempleMP}
input mpchess
beginfig(0);
init_backboard;
draw backboard;
init_chessboard;
draw chessboard;
draw_crosses(0.7[green,black])("e2","e7","c5");
endfig;
\end{ExempleMP}

\subsection{Commentaires de coup}

\mpchess permet d’afficher les classiques commentaires de coups du type
\og{}!?\fg{} grâce à la commande suivante.

\commande|draw_comment(«str»,«pos»)|\index{draw_comment@\lstinline+draw_comment+}\smallskip

\begin{description}
\item[\meta{str}:] est une chaîne de caractères (entre double-quotes) à
afficher. 
\item[\meta{pos}:] est une chaîne de caractères (entre double-quotes) constituée
d’une coordonnée (lettre et chiffre). 
\end{description}

\begin{ExempleMP}
input mpchess;
string pgnstr;
pgnstr := "1. e4 e5 2. Nf3 Nc6 3. Nxe5 Nxe5 4. Bb5 c6";
build_chessboards_from_pgn(pgnstr);
beginfig(0);
init_backboard;
draw backboard;
show_last_move(3);
draw chessboard_step(3); % Nf3
draw_comment("?!","f3");
endfig;
\end{ExempleMP}




La couleur des annotation de commentaires peut être changé grâce à la commande
suivante.
\commande|set_comment_color(«color»)|\index{set_comment_color@\lstinline+set_comment_color+}\smallskip

\subsection{Lignes principales}

\mpchess fournit une commande permettant d’afficher les flèches des coups des
lignes principales d’analyses. Il y a les commandes pour les deux couleurs. 

\commande|draw_white_main_lines(«move1»,«move2»,etc.)|\index{draw_white_main_lines@\lstinline+draw_white_main_lines+}\smallskip

\commande|draw_black_main_lines(«move1»,«move2»,etc.)|\index{draw_black_main_lines@\lstinline+draw_black_main_lines+}\smallskip

\begin{description}
\item[\meta{move1}, \meta{move2}, etc.:] sont les coups à illustrer par une
flèche, en suivant la notation de type \textsc{pgn}.
\end{description}
Lorsqu’on utilise la lecture de format \textsc{pgn} pour la construction des
positions à afficher, on pourra alors utiliser les commandes suivantes
permettant de spécifier quelle étape de la partie on commente. 

\commande|draw_white_main_lines_step(«step»)(«move1»,«move2»,etc.)|\index{draw_white_main_lines_step@\lstinline+draw_white_main_lines_step+}\smallskip

\commande|draw_black_main_lines_step(«step»)(«move1»,«move2»,etc.)|\index{draw_black_main_lines_step@\lstinline+draw_black_main_lines_step+}\smallskip

\begin{description}
  \item[\meta{step}:] est l’étape de la partie qu’on souhaite annoter; 
\item[\meta{move1}, \meta{move2}, etc.:] sont les coups à illustrer par une
flèche, en suivant la notation de type \textsc{pgn}.
\end{description}

L’exemple suivant permet d’illustrer l’utilisation de cette commande. 

\begin{ExempleMP}
input mpchess
string pgnstr;
pgnstr:="1. e4 d5";
build_chessboards_from_pgn(pgnstr);
beginfig(0);
init_backboard;
draw backboard;
draw chessboard_step(2);
draw_white_main_lines_step(2)("exd5","e5","Nc3");
endfig;
\end{ExempleMP}

Pour changer la couleur (par défaut \lstinline+0.3[_blackColorSquare,black]+),
on utilisera la commande suivante. 

\commande|set_main_lines_color(«color»)|\index{set_main_lines_color@\lstinline+set_main_lines_color+}\smallskip

\subsection{Mouvements possibles}\label{sec:possiblemoves}

\mpchess permet, à la manière du site \href{Lichess.org}{https://lichess.org},
de montrer les mouvements possibles pour une pièce. Pour cela, on utilisera la
commande suivante.

\commande|show_possible_moves(«square»)|\index{show_possible_moves@\lstinline+show_possible_moves+}\smallskip

\begin{description}
\item[\meta{square}:] est une chaîne de caractères indiquant au format classique
les coordonnées de la case où la pièce se situe. 
\end{description}

Cette commande doit être utilisée après avoir tracé le \lstinline+chessboard+. 

\begin{ExempleMP}
  input mpchess
  string fenstr;
  fenstr:="7r/2p1kp1p/p1B2p2/1pb5/8/2PP4/PP1N1PPP/R5K1 b - - 2 19";
  build_chessboard_from_fen(fenstr);
  beginfig(0);
  set_black_view;
  init_backboard;
  draw backboard;
  draw chessboard;
  show_possible_moves("c5");
  endfig;
\end{ExempleMP}  

Dans le cas où plusieurs positions ont été construites avec le format
\textsc{pgn}, on utilisera la commande suivante pour acceder à la position
choisie. 

\commande|show_possible_moves_step(«step»)(«square»)|\index{show_possible_moves_step@\lstinline+show_possible_moves_step+}\smallskip

\begin{description}
  \item[\meta{step}:] est l’étape de la partie qu’on souhaite annoter; 
\item[\meta{square}:] est une chaîne de caractères indiquant au format classique
les coordonnées de la case où la pièce se situe. 
\end{description}

Par défaut, la couleur est réglée à \lstinline+0.4[green,black]+, mais celle-ci
peut-être modifiée avec la commande suivante. 

\commande|set_possible_moves_color(«color»)|\index{set_possible_moves_color@\lstinline+set_possible_moves_color+}\smallskip


\section{Divers}

\subsection{Réinitialisation du \lstinline+chessboard+}

Pour réinitialiser la structure interne stockant les positions des pièces, on
pourra utiliser la commande suivante.

\commande|clear_chessboard|\index{clear_chessboard@\lstinline+clear_chessboard+}\smallskip


\subsection{Réinitialisation globale}

Pour réinitialiser les valeurs des différents paramètres de \mpchess, on pourra
utiliser la commande suivante.

\commande|reset_mpchess|\index{reset_mpchess@\lstinline+reset_mpchess+}\smallskip


\subsection{Découpe de l’échiquier}

On pourra faire une découpe dans l’image de l’échiquier grâce à la commande suivante.

\commande|clip_chessboard(«string»)|\index{clip_chessboard@\lstinline+clip_chessboard+}\smallskip


\begin{description}
\item[\meta{string}:] est une chaîne de caractères (entre double-quotes)
composée de deux coordonnées (lettre et chiffre) séparées par un tiret, par
exemple \lstinline+"a1-c6"+.
\end{description}

Voici un exemple d’illustration.
\begin{ExempleMP}
input mpchess;
string pgnstr;
pgnstr := "1. e4 e5 2. Nf3 Nc6 3. Nxe5 Nxe5 4. Bb5 c6";
build_chessboards_from_pgn(pgnstr);
beginfig(0);
set_black_view;
init_backboard;
draw backboard;
show_last_move(3);
draw chessboard_step(3); % Nf3
draw_comment("?!","f3");
clip_chessboard("e1-g4");
endfig;
\end{ExempleMP}




\section{Utilisation avec \LaTeX{}}

\subsection{Utilisation avec \hologo{pdfLaTeX} ou \hologo{XeLaTeX}}

Il existe plusieurs façons d’inclure les images produites par \mpchess dans un
document \LaTeX. La première est de générer des fichiers \pdf{} avec \MP{} puis
de les inclure avec \lstinline[language=TeX]+\includegraphics+. Cette
solution fonctionne avec tous les moteurs. 

On pourra aussi utiliser les packages \package{gmp} ou \package{mpgraphics}
avec \hologo{pdfLaTeX} ou \hologo{XeLaTeX}\footnote{Nous tenons à remercier
Quark67 pour les questions et les conseils.}.

\subsubsection{Avec \package{mpgraphics}}

Avec \package{mpgraphics}~\cite{ctan-mpgraphics}, on chargera \mpchess avec
l’environnement \lstinline+mpdefs+ et on pourra produire des images grâce à du
code \MP{} mais sans avoir recours à \lstinline+beginfig+ et \lstinline+endfig+,
le code pour générer une figure \MP{} se trouvant dans l’environnement
\lstinline+mpdisplay+. Il faudra de plus utiliser l’option
\lstinline+-shell-escape+ à la compilation du document \LaTeX. 

Voici un exemple complet d’illustration.

\begin{latexcode}
\documentclass{article}
\usepackage{mpgraphics}
\begin{document}
\begin{mpdefs}
input mpchess
\end{mpdefs}
\begin{mpdisplay}
init_backboard;
draw backboard;
init_chessboard;
draw chessboard;
draw_arrows(red)("e7--e5","g1|-f3");
\end{mpdisplay}
\end{document}
\end{latexcode}

\subsubsection{Avec \package{gmp}}

L’utilisation de \package{gmp}~\cite{ctan-gmp} est assez similaire à celle de
\package{mpgraphics}. Quelques commandes sont toutefois différentes, mais comme
avec \package{mpgraphics}, on n’aura pas recours à \lstinline+beginfig+ et
\lstinline+endfig+. Le chargement de \mpchess peut se faire au chargement du
package, et le code \MP{} se trouve dans l’environnement \lstinline+mpost+. Là
encore il faudra compiler le document \LaTeX{} avec l’option \lstinline+-shell-escape+.

Voici un exemple complet d’illustration. 

\begin{latexcode}
\documentclass{article}
\usepackage[shellescape, everymp={input mpchess;}]{gmp}

\begin{document}

\begin{mpost}
init_backboard;
draw backboard;
init_chessboard;
draw chessboard;
draw_arrows(red)("e7--e5","g1|-f3");
\end{mpost}
\end{document}
\end{latexcode}

\subsection{Utilisation avec \hologo{LuaLaTeX} et \package{luamplib}}

Il est tout à fait possible d’utiliser \mpchess directement dans un fichier
\LaTeX{} avec \hologo{LuaLaTeX}\footnote{Rappelons que \MP{} fait partie
intégrante de \hologo{LuaTeX}.} et le package \package{luamplib}. C’est
d’ailleurs ce qui est fait pour écrire cette documentation.

Il suffira alors de mettre le code \MP{} dans l’environnement
\lstinline+mplibcode+ comme illustré plus bas.
  

\mpchess utilise, pour certaines fonctionnalités, l’opérateur
\lstinline+infont+ de  \MP. Ainsi, pour que le contenu de ces fonctionnalités
soit composé dans la fonte courante du document, on devra ajouter dans son
document
\LaTeX{}, la commande :
\begin{latexcode}
\mplibtextextlabel{enable}
\end{latexcode}

Pour plus de détails sur ces mécanismes, nous renvoyons à la documentation du
package \package{luamplib}~\cite{ctan-luamplib}.

On pourra charger globaletement \mpchess avec la commande suivante. 

\begin{latexcode}
\everymplib{input mpchess;}
\end{latexcode}

Voici un exemple complet d’illustration (à compiler avec \hologo{LuaLaTeX}). 

\begin{latexcode}
\documentclass{article}
\usepackage{luamplib}

\everymplib{input mpchess;}

\begin{document}

\begin{mplibcode}
beginfig(0);
init_backboard;
draw backboard;
init_chessboard;
draw chessboard;
draw_arrows(red)("e7--e5","g1|-f3");
endfig;
\end{mplibcode}
\end{document}
\end{latexcode}

\subsection{Font TrueType}

Le package \mpchess{} fournit une fonte très simple composé d’uniquement les 12
glyphes correspondant aux pièces noires et blanches dans la table Unicode. Pour
y avoir accès il suffit d’utiliser \hologo{LuaLaTeX} ou \hologo{XeLaTeX} et le
package \package{fontspec}.  Pour faciliter son utilisation, nous proposons
quelques définitions.

\begin{latexcode}
\documentclass{article}
\usepackage{fontspec}
\newfontfamily{\chessfont}{mpchess font}
\newcommand\bP{{\chessfont \char"265F}} % black Pawn
\newcommand\bN{{\chessfont \char"265E}} % black Knight
\newcommand\bB{{\chessfont \char"265D}} % black Bishop
\newcommand\bR{{\chessfont \char"265C}} % black Rook
\newcommand\bQ{{\chessfont \char"265B}} % black Queen
\newcommand\bK{{\chessfont \char"265A}} % black King
\newcommand\wP{{\chessfont \char"2659}} % white Pawn
\newcommand\wN{{\chessfont \char"2658}} % white Knight
\newcommand\wB{{\chessfont \char"2657}} % white Bishop
\newcommand\wR{{\chessfont \char"2656}} % white Rook
\newcommand\wQ{{\chessfont \char"2655}} % white Queen
\newcommand\wK{{\chessfont \char"2654}} % white King

\begin{document}

Voici l'enchainement de coups : 1. e4 e5 2. \wB c4 d6 3. \wN f3 \bB g4 4. \wN
c3 g6 5. \wN xe5 \bB xd1.

\end{document}
\end{latexcode}

Qui produira :

\og{}Voici l’enchainement de coups : 1. e4 e5 2. \wB c4 d6 3. \wN f3 \bB g4
4. \wN c3 g6 5. \wN xe5 \bB xd1.\fg{}



\section{To do}

De nombreuses choses sont à ajouter à \mpchess. Parmi celles-ci, on peut penser
à :

\begin{itemize}
\item afficher les pièces capturées, ou le différentiel des pièces capturées ;
\item afficher le temps restant pour chaque joueur ;
\item parser le format \textsc{pgn} complet et donc avec les tags optionnels ;
\item ajouter les coordonnées en extérieur lorsque le plateau est découpé ;
\item ajouter des thèmes de pièces.
\end{itemize}

\section{Un exemple complet}

\begin{ExempleMP}[sidebyside=false]
input mpchess
string pgnstr;
pgnstr:="1. e4 e5 2. Bc4 d6 3. Nf3 Bg4 4. Nc3 g6 5. Nxe5 Bxd1";
build_chessboards_from_pgn(pgnstr);
beginfig(0);
set_backboard_width(8cm);
set_white_player("Kermur de Legal");
set_black_player("Saint-Brie");
init_backboard;
draw backboard;
show_last_move(10);
draw_comment("?","d1");
color_square(0.3[green,black])("c4","c3","e5");
color_square(0.3[red,black])("e8");
draw chessboard_step(10);
draw_arrows(0.3[green,black])("e5|-f7","c3-|d5");
draw_arrows(0.3[red,black])("c4--f7");
endfig;
\end{ExempleMP}

\section{Historique}

\begin{description}
\item[v0.7, juillet 2023:] Ajustement du cavalier noir, ajout de thèmes
d'échiquier (\lstinline+GreenLichess+, \lstinline+PinkPyramidalLichess+,
\lstinline+Wood+ avec des types de coloriage (\lstinline+set_board_type+). 
\item[v0.6, avril 2023:] Corrections de bugs concernant la gestion des roques, et
des ambiguïtés de pièces pour les déplacements sous format \textsc{pgn}.
Changement de \lstinline+set_white_color+ en
\lstinline+set_white_squares_color+ et \lstinline+set_black_color+ en
\lstinline+set_black_squares_color+. Changement de \lstinline+set_no_coords+ en
\lstinline+hide_coordinates+ et \lstinline+set_coords+ en
\lstinline+show_coordinates+. Changement de \lstinline+set_whos_to_move+ en
\lstinline+show_whos_to_move+ et \lstinline+unset_whos_to_move+ en
\lstinline+hide_whos_to_move+. Changement de \lstinline+set_coords_inside+ et
\lstinline+set_coords_outside+ en \lstinline+set_coordinates_position+.
Changement de \lstinline+set_coords_font+ en \lstinline+set_coordinates_font+. 
\item[v0.5, 20 avril 2023:] Correction d’un bug, \textbf{changement de
l’ensemble de pièce par défaut} pour l’ensemble \lstinline+mpchess+ (qui a été
ajouté à Lichess), ajout de la fonte TrueType, et mise à jour de la documentation.
\item[v0.4, 6 avril 2023:] Corrections dans la documentation, notamment la
version anglaise; ajout des commandes pour visualiser les mouvements possible
pour une pièce (section~\ref{sec:possiblemoves}). 
\item[v0.3, 29 mars 2023 :] Petit bug.
\item[v0.2, 28 mars 2023 :] Ajout des commandes de lecture de fichiers
\textsc{pgn} et \textsc{fen}; ajout des commandes d’affichage des lignes
principales d’analyse; suppression du thème \lstinline+staunty+ (pour cause de
licence) et création du thème de pièces \lstinline+mpchess+.
\item[v0.1, 23 mars 2023 :] Première publication sur le \ctan.
\end{description}


\section{Remerciements}
Nous souhaitons remercier Quark67 pour ses retours et ses corrections, Douglas
Johnson pour avoir corrigé la version anglaise de la documentation et Hans
Nieuwenhuis pour ses conseils. 
Ces retours et encouragements font extrêmement plaisir !


  \printbibliography
\printindex
\end{document}



%%% Local Variables:
%%% flyspell-mode: 1
%%% ispell-local-dictionary: "fr"
%%% End:
