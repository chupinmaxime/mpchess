% mpchess: draw boards of chess positions with MetaPost
%
% Originally written by Maxime Chupin <notezik@gmail.com>,
% 2023.
%
% Distributed under the terms of the GNU free documentation licence:
%   http://www.gnu.org/licenses/fdl.html
% without any invariant section or cover text.

\documentclass[french]{ltxdoc}
\usepackage{tcolorbox}
\tcbuselibrary{listings,breakable}
%\tcbuselibrary{documentation}
\usepackage{enumitem}
\usepackage[tikz]{bclogo}
\usepackage{mflogo}
\usepackage{hologo}
\usepackage{luamplib}
\mplibtextextlabel{enable}
\usepackage{biblatex}
\usepackage{wrapfig}
\usepackage{fancyvrb,xparse,xargs}
\usepackage[sfdefault]{FiraSans}
\usepackage[mathrm=sym]{firamath-otf}
%\setmonofont{Fira Mono}
\setmonofont{FiraCode-Regular.ttf}[BoldFont= FiraCode-Bold.ttf,ItalicFont= FiraCode-RegularItalic.otf,BoldItalicFont= FiraCode-BoldItalic.otf,]

\usepackage{xspace}
\usepackage{animate}
\usepackage{babel}
\newcommand{\ctan}{\textsc{ctan}}
\NewDocumentCommand{\package}{ m }{%
  \href{https://ctan.org/pkg/#1}{#1}\xspace
}

\definecolor{darkred}{rgb}{0.8,0.1,0.1}
\definecolor{vert}{rgb}{0.1,0.4,0.1}
\definecolor{bleu}{rgb}{0.2,0.2,0.6}
\definecolor{orange}{rgb}{0.6,0.4,0.}
\colorlet{code}{blue!80!black}

\usepackage[colorlinks=true,urlcolor=orange,linkcolor=orange,menucolor=black]{hyperref}

\newcommand \file       {\nolinkurl}
\renewcommand \cmd        {\texttt}
\renewcommand \code   [1] {\texorpdfstring {\texttt{\color{code}#1}} {#1}}
\renewcommand*\cs     [1] {\code{\textbackslash #1}}



\newcommand*\commande{\noindent\hspace{-30pt}%
  \SaveVerb[aftersave={%
   \UseVerb{Vitem}
  }%
  ]{Vitem}}

\newcommand*\textme[1]{\textcolor{black}{\rmfamily\textit{#1}}}
%\renewcommand*\meta[1]{% % meta
%  \textme{\ensuremath{\langle}#1\ensuremath{\rangle}}}
\newcommand*\optstar{% % optional star
  \meta{\ensuremath{*}}\xspace}
\DefineShortVerb{\|}
\newcommand\R{\mathbf{R}}
\setlength{\fboxsep}{2pt}
\fvset{%
  codes={\catcode`\«\active \catcode`\×\active },
  defineactive={\makefancyog\makefancytimes},
  formatcom=\color{darkred},
  frame=single
}
% rendre «...» équivalent à \meta{...}
{\catcode`\«\active
  \newcommandx\makefancyog[0][addprefix=\global]{%
    \def«##1»{\meta{##1}}}}
% rendre × équivalent à \optstar
{\catcode`\×\active
  \newcommandx\makefancytimes[0][addprefix=\global]{%
    \def×{\optstar{}}}}


\newcommand\mpchess{\textbf{\textlogo{MP}}\textit{chess}\xspace}



%\addbibresource{biblio.bib}


\lstset{
  numberstyle=\footnotesize\color{vert},
  keywordstyle=\ttfamily\bfseries\color{bleu},
  basicstyle=\ttfamily,
  commentstyle=\itshape\color{vert},
  stringstyle=\ttfamily,
  showstringspaces=false,
  language=MetaPost,
  breaklines=true,
  breakindent=30pt,
  defaultdialect=MetaPost,
  morekeywords={}% frame=tb
}


\newtcblisting{Exemple}{%
  arc=0pt,outer arc=0pt,
  colback=red!2!white,
  colframe=red!75!black,
  breakable,
  boxsep=0pt,left=5pt,right=5pt,top=5pt,bottom=5pt, bottomtitle =
  3pt, toptitle=3pt,
  boxrule=0pt,bottomrule=0.5pt,toprule=0.5pt, toprule at break =
  0pt, bottomrule at break = 0pt,
  listing options={breaklines},
}

\newtcblisting{commandshell}{colback=black,colupper=white,colframe=black,
  arc=0pt,
  listing only,boxsep=0pt,listing
  options={style=tcblatex,language=sh},
  every listing line={\textcolor{red}{\small\ttfamily\bfseries user \$> }}}


  \newtcblisting{mpcode}{
  arc=0pt,outer arc=0pt,
  colback=red!2!white,
  colframe=red!75!black,
  breakable,
  boxsep=0pt,left=5pt,right=5pt,top=5pt,bottom=5pt, bottomtitle =
  3pt, toptitle=3pt,
  boxrule=0pt,bottomrule=0.5pt,toprule=0.5pt, toprule at break =
  0pt, bottomrule at break = 0pt,
  listing only,boxsep=0pt,listing
  options={breaklines}
}

\newtcblisting{latexcode}{
  arc=0pt,outer arc=0pt,
  colback=red!2!white,
  colframe=red!75!black,
  breakable,
  boxsep=0pt,left=5pt,right=5pt,top=5pt,bottom=5pt, bottomtitle =
  3pt, toptitle=3pt,
  boxrule=0pt,bottomrule=0.5pt,toprule=0.5pt, toprule at break =
  0pt, bottomrule at break = 0pt,
  listing only,boxsep=0pt,listing
  options={breaklines,language={[LaTeX]TeX}}
}





%\lstset{moredelim=*[s][\color{red}\rmfamily\itshape]{<}{>}}
%\lstset{moredelim=*[s][\color{blue}\rmfamily\itshape]{<<}{>>}}

\begin{document}

\title{{MPchess} : dessiner des plateaux d’échecs et des positions avec \hologo{METAPOST}}
\author{Maxime Chupin, \url{notezik@gmail.com}}
\date{\today}

%% === Page de garde ===================================================
\thispagestyle{empty}
\begin{tikzpicture}[remember picture, overlay]
  \node[below right, shift={(-4pt,4pt)}] at (current page.north west) {%
    \includegraphics{fond.pdf}%
  };
\end{tikzpicture}%

\noindent
{\Huge \mpchess}\par\medskip
\noindent
{\Large  dessiner des plateaux d’échecs et des positions avec \hologo{METAPOST}}\\[1cm]
\parbox{0.6\textwidth}{
  \begin{mplibcode}
  input mpchess

  string in;
  in:="1. e4 e5 2. Bc4 d6 3. Nf3 Bg4 4. Nc3 g6 5. Nxe5 Bxd1 ";
  build_chessboards_from_pgn(in);
  
  beginfig(0);
  set_backboard_width(8cm);
  init_backboard;
  draw backboard;
  show_last_move(10);
  color_square(0.3[green,black])("c4","c3","e5");
  color_square(0.3[red,black])("e8");
  draw chessboard_step(10);
  draw_arrows(0.3[green,black])("e5|-f7","c3-|d5");
  draw_arrows(0.3[red,black])("c4--f7");
  endfig;
  \end{mplibcode}
}\hfill
\parbox{0.5\textwidth}{\Large\raggedleft
  \textbf{Contributor}\\
  Maxime \textsc{Chupin}\\
  \url{notezik@gmail.com}
}
\vfill
\begin{center}
  Version 0.1, 2023, March, 8th \\
  \url{https://plmlab.math.cnrs.fr/mchupin/mpchess}
\end{center}
%% == Page de garde ====================================================
\newpage

%\maketitle

\begin{abstract}
Abstract
\end{abstract}


\begin{center}
  \url{https://plmlab.math.cnrs.fr/mchupin/mpchess}
\end{center}

\tableofcontents


\section{Installation}

\mpchess est sur le \ctan{} et peut être installé via le gestionnaire de package
de votre distribution. 

\begin{center}
  \url{https://www.ctan.org/pkg/mpchess}
\end{center}


\subsection{Avec la \TeX live sous Linux ou MacOS}

Pour installer \mpchess avec \TeX live, il vous faudra créer le répertoire 
\lstinline+texmf+ dans votre \lstinline+home+.

\begin{commandshell}
mkdir ~/texmf
\end{commandshell}

Ensuite, il faudra y placer les fichiers \lstinline+.mp+ dans le répertoire \begin{center}
  \lstinline+~/texmf/tex/metapost/mpchess/+
\end{center}

\mpchess est constitué de 5 fichiers \hologo{METAPOST} :
\begin{itemize}
  \item \lstinline+mpchess.mp+;
  \item \lstinline+mpchess-chessboard.mp+;
  \item \lstinline+mpchess-pgn.mp+;
  \item \lstinline+mpchess-fen.mp+;
  \item \lstinline+mpchess-cburnett.mp+.
\end{itemize}

Une fois fait cela, \mpchess sera chargé avec le classique
\begin{mpcode}
input mpchess
\end{mpcode}

\subsection{Avec Mik\TeX{} et Windows}

Ces deux systèmes sont inconnus de l’auteur de \mpchess, ainsi, nous renvoyons à
leurs documentations pour y ajouter des packages locaux:
\begin{center}
  \url{http://docs.miktex.org/manual/localadditions.html}
\end{center}



\subsection{Dépendances}

\mpchess dépend des packages \hologo{METAPOST}: \package{hatching} et, si
\mpchess n’est pas utilisé avec \hologo{LuaLaTeX} et \package{luamplib},
\package{latexmp}.

\subsection{Utilisation avec \hologo{LuaLaTeX} et \package{luamplib}}

Il est tout à fait possible d’utiliser \mpchess directement dans un fichier
\LaTeX{} avec \hologo{LuaLaTeX} et le package \package{luamplib}. C’est
d’ailleurs ce qui est fait pour écrire cette documentation.  

\mpchess utilise, pour certaines fonctionnalités, l’opérateur
\lstinline+infont+ de  \hologo{METAPOST}. Ainsi, pour que le contenu de ces fonctionnalités soient
composé dans la fonte courante du document, on devra ajouter dans son document
\LaTeX{}, la commande :
\begin{latexcode}
\mplibtextextlabel{enable}
\end{latexcode}

Pour plus de détails sur ces mécanismes, nous renvoyons à la documentation du
package \package{luamplib}~\cite{luamplib}. 
\section{Philosophie générale}

Avec \mpchess, on construit l’image finale du plateau d’échec avec les pièces
par couches successives. Ainsi, on commencera par produire et dessiner le
plateau (\lstinline+backboard+), que l’on pourra modifier en colorant par
exemple certaine cases, ensuite on ajoutera les pièces de la position
(\lstinline+chessboard+), et enfin, on pourra annoter le tout avec des marques, des couleurs, des flêches, etc. 

Par ailleurs, \mpchess produit des images proches graphiquement de ce que peut
fournir l’excellent site \emph{open source} \url{https://lichess.org}. Vous
verrez que les couleurs, les pièces, et l’aspect général sont largement inspirés
de ce que propose ce site. 

\section{Plateau}

\subsection{Réglage des tailles}

\subsection{Nombre de case}

\subsection{Réglage du thème de couleur}

\subsection{Affichage des coordonnées}

\subsection{Vue blanche ou noire}

\subsection{Noms des joueurs}


\section{Pièces et positions}

\subsection{Réglage du thème des pièces}

\subsection{Lecture de données au format \textsc{pgn}}

\subsection{Lecture de données au format \textsc{fen}}


\section{Annotation}


\commande|test_initbackboard«test»|


\printbibliography

\end{document}



%%% Local Variables:
%%% flyspell-mode: 1
%%% ispell-local-dictionary: "american"
%%% End:
