% mpchess: draw boards of chess positions with MetaPost
%
% Originally written by Maxime Chupin <notezik@gmail.com>,
% 2023.
%
% Distributed under the terms of the GNU free documentation licence:
%   http://www.gnu.org/licenses/fdl.html
% without any invariant section or cover text.

\documentclass[french]{ltxdoc}
\usepackage{tcolorbox}
\tcbuselibrary{listings,breakable}
%\tcbuselibrary{documentation}
\usepackage{enumitem}
\usepackage[tikz]{bclogo}
\usepackage{mflogo}
\usepackage{hologo}
\usepackage{luamplib}
\mplibtextextlabel{enable}
\usepackage{biblatex}
\addbibresource{ctan.bib}
\usepackage{wrapfig}
\usepackage{siunitx}
\usepackage{makeidx}
\usepackage{fancyvrb,xparse,xargs}
\usepackage[sfdefault]{FiraSans}
\usepackage[mathrm=sym]{firamath-otf}
%\setmonofont{Fira Mono}
\setmonofont{FiraCode-Regular.ttf}[BoldFont= FiraCode-Bold.ttf,ItalicFont= FiraCode-RegularItalic.otf,BoldItalicFont= FiraCode-BoldItalic.otf,]

\usepackage{xspace}
\usepackage{animate}
\usepackage{babel}
\newcommand{\ctan}{\textsc{ctan}}
\NewDocumentCommand{\package}{ m }{%
  \href{https://ctan.org/pkg/#1}{#1}\xspace
}

\definecolor{darkred}{rgb}{0.6,0.1,0.1}
\definecolor{vert}{rgb}{0.1,0.4,0.1}
\definecolor{bleu}{rgb}{0.2,0.2,0.6}
\definecolor{orange}{rgb}{0.6,0.4,0.}
\colorlet{code}{blue!80!black}

\usepackage[colorlinks=true,urlcolor=orange,linkcolor=orange,menucolor=black,citecolor=orange]{hyperref}

\newcommand \file       {\nolinkurl}
\renewcommand \cmd        {\texttt}
\renewcommand \code   [1] {\texorpdfstring {\texttt{\color{code}#1}} {#1}}
\renewcommand*\cs     [1] {\code{\textbackslash #1}}



\newcommand*\commande{\par\bigskip\noindent\hspace{-30pt}%
  \SaveVerb[aftersave={%
   \UseVerb{Vitem}%
  }%
  ]{Vitem}%
  }

\newcommand*\textme[1]{\textcolor{black}{\rmfamily\textit{#1}}}
%\renewcommand*\meta[1]{% % meta
%  \textme{\ensuremath{\langle}#1\ensuremath{\rangle}}}
\newcommand*\optstar{% % optional star
  \meta{\ensuremath{*}}\xspace}
\DefineShortVerb{\|}
\newcommand\R{\mathbf{R}}
\setlength{\fboxsep}{2pt}
\fvset{%
  codes={\catcode`\«\active \catcode`\×\active },
  defineactive={\makefancyog\makefancytimes},
  formatcom=\color{darkred},
  frame=single
}
% rendre «...» équivalent à \meta{...}
{\catcode`\«\active
  \newcommandx\makefancyog[0][addprefix=\global]{%
    \def«##1»{\meta{##1}}}}
% rendre × équivalent à \optstar
{\catcode`\×\active
  \newcommandx\makefancytimes[0][addprefix=\global]{%
    \def×{\optstar{}}}}


\newcommand\mpchess{\textbf{\textlogo{MP}}\textit{chess}\xspace}



%\addbibresource{biblio.bib}


\lstset{
  numberstyle=\footnotesize\color{vert},
  keywordstyle=\ttfamily\bfseries\color{bleu},
  basicstyle=\ttfamily,
  commentstyle=\itshape\color{vert},
  stringstyle=\ttfamily\color{orange},
  showstringspaces=false,
  language=MetaPost,
  breaklines=true,
  breakindent=30pt,
  defaultdialect=MetaPost,
  classoffset=1,% frame=tb
  morekeywords={init_backboard,set_backboard_width,set_backboard_size,set_color_theme,get_backboard_width,get_backboard_size,set_backboard_width,get_square_dim,set_white_color,set_black_color,set_coords_inside,set_coords_outside,set_coords_font,set_coords, set_no_coords},
  keywordstyle=\color{darkred},
  classoffset=2,% frame=tb
  morekeywords={backboard,chessboard},
  keywordstyle=\color{vert},
  classoffset=0,% frame=tb
  morekeywords={draw},
  keywordstyle=\color{bleu}
}

\makeatletter
\tcbset{%
    listing metapost/.code={%
        \def\tcbuselistingtext@input{\begin{mplibcode} input \jobname.listing; \end{mplibcode}}%
    }
}
\makeatother
\newtcblisting[auto counter,]{ExempleMP}[1][]{%
  arc=0pt,outer arc=0pt,
  colback=darkred!3,
  colframe=darkred,
  breakable,
  boxsep=0pt,left=5pt,right=5pt,top=5pt,bottom=5pt, bottomtitle =
  3pt, toptitle=3pt,
  boxrule=0pt,bottomrule=0.5pt,toprule=0.5pt, toprule at break =
  0pt, bottomrule at break = 0pt,
  listing side text,
  listing metapost, 
  title=Exemple~\thetcbcounter,
  listing options={breaklines},#1
}

\newtcblisting{commandshell}{colback=black,colupper=white,colframe=black,
  arc=0pt,
  listing only,boxsep=0pt,listing
  options={style=tcblatex,language=sh},
  every listing line={\textcolor{red}{\small\ttfamily\bfseries user \$> }}}


  \newtcblisting{mpcode}{
  arc=0pt,outer arc=0pt,
  colback=darkred!3,
  colframe=darkred,
  breakable,
  boxsep=0pt,left=5pt,right=5pt,top=5pt,bottom=5pt, bottomtitle =
  3pt, toptitle=3pt,
  boxrule=0pt,bottomrule=0.5pt,toprule=0.5pt, toprule at break =
  0pt, bottomrule at break = 0pt,
  listing only,boxsep=0pt,listing
  options={breaklines}
}

\newtcblisting{latexcode}{
  arc=0pt,outer arc=0pt,
  colback=darkred!3,
  colframe=darkred,
  breakable,
  boxsep=0pt,left=5pt,right=5pt,top=5pt,bottom=5pt, bottomtitle =
  3pt, toptitle=3pt,
  boxrule=0pt,bottomrule=0.5pt,toprule=0.5pt, toprule at break =
  0pt, bottomrule at break = 0pt,
  listing only,boxsep=0pt,listing
  options={breaklines,language={[LaTeX]TeX}}
}


\makeindex



%\lstset{moredelim=*[s][\color{red}\rmfamily\itshape]{<}{>}}
%\lstset{moredelim=*[s][\color{blue}\rmfamily\itshape]{<<}{>>}}

\begin{document}

\title{{MPchess} : dessiner des plateaux d’échecs et des positions avec \hologo{METAPOST}}
\author{Maxime Chupin, \url{notezik@gmail.com}}
\date{\today}

%% === Page de garde ===================================================
\thispagestyle{empty}
\begin{tikzpicture}[remember picture, overlay]
  \node[below right, shift={(-4pt,4pt)}] at (current page.north west) {%
    \includegraphics{fond.pdf}%
  };
\end{tikzpicture}%

\noindent
{\Huge \mpchess}\par\medskip
\noindent
{\Large  dessiner des plateaux d’échecs et des positions avec \hologo{METAPOST}}\\[1cm]
\parbox{0.6\textwidth}{
  \begin{mplibcode}
  input mpchess

  string in;
  in:="1. e4 e5 2. Bc4 d6 3. Nf3 Bg4 4. Nc3 g6 5. Nxe5 Bxd1 ";
  build_chessboards_from_pgn(in);
  
  beginfig(0);
  set_backboard_width(8cm);
  init_backboard;
  draw backboard;
  show_last_move(10);
  color_square(0.3[green,black])("c4","c3","e5");
  color_square(0.3[red,black])("e8");
  draw chessboard_step(10);
  draw_arrows(0.3[green,black])("e5|-f7","c3-|d5");
  draw_arrows(0.3[red,black])("c4--f7");
  endfig;
  \end{mplibcode}
}\hfill
\parbox{0.5\textwidth}{\Large\raggedleft
  \textbf{Contributor}\\
  Maxime \textsc{Chupin}\\
  \url{notezik@gmail.com}
}
\vfill
\begin{center}
  Version 0.1, 2023, March, 8th \\
  \url{https://plmlab.math.cnrs.fr/mchupin/mpchess}
\end{center}
%% == Page de garde ====================================================
\newpage

%\maketitle

\begin{abstract}
Abstract
\end{abstract}


\begin{center}
  \url{https://plmlab.math.cnrs.fr/mchupin/mpchess}
\end{center}

\tableofcontents


\section{Installation}

\mpchess est sur le \ctan{} et peut être installé via le gestionnaire de package
de votre distribution. 

\begin{center}
  \url{https://www.ctan.org/pkg/mpchess}
\end{center}


\subsection{Avec la \TeX live sous Linux ou MacOS}

Pour installer \mpchess avec \TeX live, il vous faudra créer le répertoire 
\lstinline+texmf+ dans votre \lstinline+home+.

\begin{commandshell}
mkdir ~/texmf
\end{commandshell}

Ensuite, il faudra y placer les fichiers \lstinline+.mp+ dans le répertoire \begin{center}
  \lstinline+~/texmf/tex/metapost/mpchess/+
\end{center}

\mpchess est constitué de 5 fichiers \hologo{METAPOST} :
\begin{itemize}
  \item \lstinline+mpchess.mp+;
  \item \lstinline+mpchess-chessboard.mp+;
  \item \lstinline+mpchess-pgn.mp+;
  \item \lstinline+mpchess-fen.mp+;
  \item \lstinline+mpchess-cburnett.mp+.
\end{itemize}

Une fois fait cela, \mpchess sera chargé avec le classique
\begin{mpcode}
input mpchess
\end{mpcode}

\subsection{Avec Mik\TeX{} et Windows}

Ces deux systèmes sont inconnus de l’auteur de \mpchess, ainsi, nous renvoyons à
leurs documentations pour y ajouter des packages locaux:
\begin{center}
  \url{http://docs.miktex.org/manual/localadditions.html}
\end{center}



\subsection{Dépendances}

\mpchess dépend des packages \MP: \package{hatching} et, si
\mpchess n’est pas utilisé avec \hologo{LuaLaTeX} et \package{luamplib},
\package{latexmp}.

\subsection{Utilisation avec \hologo{LuaLaTeX} et \package{luamplib}}

Il est tout à fait possible d’utiliser \mpchess directement dans un fichier
\LaTeX{} avec \hologo{LuaLaTeX} et le package \package{luamplib}. C’est
d’ailleurs ce qui est fait pour écrire cette documentation.  

\mpchess utilise, pour certaines fonctionnalités, l’opérateur
\lstinline+infont+ de  \MP. Ainsi, pour que le contenu de ces fonctionnalités soient
composé dans la fonte courante du document, on devra ajouter dans son document
\LaTeX{}, la commande :
\begin{latexcode}
\mplibtextextlabel{enable}
\end{latexcode}

Pour plus de détails sur ces mécanismes, nous renvoyons à la documentation du
package \package{luamplib}~\cite{ctan-luamplib}. 
\section{Pourquoi ce package et philosophie générale}

Il existe déjà des packages \LaTeX{} pour dessiner des plateaux d’échecs et des
positions dont le très bon \package{xskak}~\cite{ctan-xskak} qui couplé avec le
package \package{chessboard}~\cite{ctan-chessboard}. Ulrike Fisher a réalisé là un
travail d’amélioration, de maintient, et nous a fournit d’excellent outils pour
réaliser des diagrammes d’échecs et de traiter les différents formats de
descriptions de parties\footnote{Elle a même développé le
package~\package{chessfss} pour gérer divers fontes d’échec.}. Les
documentations de ces packages sont de très bonnes qualités. 

Plusieurs choses ont motivé la création de \mpchess. Tout d’abord, avec
\package{chessboard} l’ajout d’ensemble de pièces n’est pas très aisé car cela
repose sur des fontes. De plus, je trouve que le dessin de diagrammes de parties
d’échec est quelque chose de très graphique, et que le passage par un langage
dédié au dessin offre plus de souplesse et quoi de mieux que \MP~\cite{ctan-metapost}.  


Avec \mpchess, on construit l’image finale du plateau d’échec avec les pièces
par couches successives. Ainsi, on commencera par produire et dessiner le
plateau (\lstinline+backboard+), que l’on pourra modifier en colorant par
exemple certaine cases, ensuite on ajoutera les pièces de la position
(\lstinline+chessboard+), et enfin, on pourra annoter le tout avec des marques, des couleurs, des flêches, etc. 

Par ailleurs, \mpchess produit des images proches graphiquement de ce que peut
fournir l’excellent site \emph{open source} \url{https://lichess.org}. Vous
verrez que les couleurs, les pièces, et l’aspect général sont largement inspirés
de ce que propose ce site. 

\section{Plateau}

Le plateau est appelé avec \mpchess{} \lstinline+backboard+. 
Il faudra initialiser le plateau avant de le dessiner. Cela se fait avec la commande suivante:\par

\commande|init_backboard|\smallskip\index{init_backboard@\lstinline+init_backboard+}


Cette commande construit une \lstinline+picture+ de \MP{} nommée
\mbox{\lstinline+backboard+.} Il faudra ensuite la tracer comme l’illustre l’exemple
suivant. 

\begin{ExempleMP}
input mpchess

beginfig(0);
init_backboard;
draw backboard;
endfig;
\end{ExempleMP}

Cette initialisation permettra de prendre en compte les différentes options et
fonctionnalités que nous allons décrire dans la suite. 
\subsection{Réglage des tailles}
Lors de la création du \lstinline+backboard+, on peux décider de la largeur de celui-ci. Cela se fait grâce à la commande suivante :\par\bigskip

\commande|set_backboard_width(«dim»)|\smallskip\index{set_backboard_width@\lstinline+set_backboard_width+}

\begin{description}
  \item[\meta{dim}:] est la largeur de plateau de jeu souhaitée (avec l’unité). Par défaut, cette dimension est à \SI{5}{cm}.
\end{description}

L’utilisation de cette commande est illustré à l’exemple~\ref{ex:widthcase}. Cette commande est à utilisée avant \lstinline+init_backboard+ pour qu’elle soit prise en compte à la création de l’image. 
\bigskip

On peut récupérer la dimension du plateau de jeu par la commande suivante.

\commande|get_backboard_width|\smallskip\index{get_backboard_width@\lstinline+get_backboard_width+}

Cette commande retourne un type \lstinline+numeric+. 

\subsection{Nombre de case}
Par défaut, le plateau de jeu contient 64 cases ($8\times 8$). On peut modifier
cela avec la commande suivante:

\commande|set_backboard_size(«nbr»)|
\smallskip\index{set_backboard_size@\lstinline+set_backboard_size+}

\begin{description}
  \item[\meta{nbr}:] est le nombre de cases souhaité. Le plateau sera alors carré de taille \meta{nbr}$\times$\meta{nbr}. Par défaut ce nombre est à 8.
\end{description}

Encore une fois, cette commande est à utilisée avant \lstinline+init_backboard+ pour qu’elle soit prise en compte comme le montre l’exemple suivant. 

\begin{ExempleMP}[label=ex:widthcase]
input mpchess

beginfig(0);
set_backboard_width(3cm);
set_backboard_size(6);
init_backboard;
draw backboard;
endfig;
\end{ExempleMP}

Pour obtenir la taille du plateau de jeu par la commande suivante.

\commande|get_backboard_size|\smallskip\index{get_backboard_size@\lstinline+get_backboard_size+}

Cette commande retourne un type \lstinline+numeric+. 

\subsection{Dimension d’une case}

En fonction du nombre de cases sur le plateau et la largeur prescrite pour le plateau, \mpchess calcule la dimension (largeur ou hauteur) d’une case. Cela sert d’unité générale. Pour l’obtenir, on utilisera la commande suivante. 

\commande|get_square_dim|\smallskip\index{get_square_dim@\lstinline+get_square_dim+}

Cette commande retourne un \lstinline+numeric+. 

\subsection{Réglage du thème de couleur}

\subsubsection{Thèmes prédéfinis}

Plusieurs thèmes de couleurs sont accessibles. Pour choisir un thème de couleur, on utilisera la commande suivante :

\commande|set_color_theme(«string»)|\smallskip\index{set_color_theme@\lstinline+set_color_theme+}

\meta{string} peut valoir :
\begin{itemize}
\item \lstinline+"BlueLichess"+ (thème par défaut);
\item \lstinline+"BrownLichess"+ ;
\item ou \lstinline+"Classical"+.
\end{itemize}

Les exemples montrent les résultats obtenus.
\begin{ExempleMP}
input mpchess
beginfig(0);
set_color_theme("BrownLichess");
init_backboard;
draw backboard;
endfig;
\end{ExempleMP}
\begin{ExempleMP}
input mpchess
beginfig(0);
set_color_theme("Classical");
init_backboard;
draw backboard;
endfig;
\end{ExempleMP}

Les deux thèmes colorés fournis sont les couleurs des thèmes de Lichess. 

\subsubsection{Configuration d’un thème personnel}

Un thème de couleur est en réalité simplement la définition de deux couleurs. 
Celles-ci peuvent se définir avec les commandes suivante.

\commande|set_white_color(«color»)|\index{set_white_color@\lstinline+set_white_color+}\par
\commande|set_black_color(«color»)|\index{set_black_color@\lstinline+set_black_color+}\smallskip

\meta{color} est une \lstinline+color+ \MP. 

\begin{ExempleMP}
input mpchess
beginfig(0);
set_white_color((0.9,0.8,0.8));
set_black_color((0.7,0.6,0.6));
init_backboard;
draw backboard;
endfig;
\end{ExempleMP}
\subsection{Affichage des coordonnées}

Vous avez pu constater dans les divers exemples que par défaut, les coordonnées sont, comme le fait le site Lichess, inscrites en petit à l’intérieur des cases. 

\mpchess permet de positionner ces coordonnées à l’extérieur du plateau avec la commande suivante.

\commande|set_coords_outside|\index{set_coords_outside@\lstinline+set_coords_outside+}\smallskip
Le résultat est alors le suivante. 
\begin{ExempleMP}
input mpchess
beginfig(0);
set_coords_outside;
init_backboard;
draw backboard;
endfig;
\end{ExempleMP}

Il existe aussi la commande permettant de positionner les coordonnées à l’intérieur du plateau.

\commande|set_coords_inside|\index{set_coords_inside@\lstinline+set_coords_inside+}\smallskip



Vous pouvez constater dans cette documentation qu’avec \package{luamplib} et
\LaTeX, la fonte est la fonte du document courant. Pour tracer ces lettres et
ces chiffres, \mpchess utilise l’opérateur \MP{} \lstinline+infont+ et la fonte
est réglée à \lstinline+defaultfont+ par défaut. On peut modifier cette fonte avec la commande suivante. 

\commande|set_coords_font(«font»)|\index{set_coords_font@\lstinline+set_coords_font+}\smallskip

Il faudra alors utiliser les conventions de nommage propres à l’opérateur \lstinline+infont+ de \MP{} et nous renvoyons à la documentation~\cite{ctan-metapost} pour plus de détails. 

On pourra aussi supprimer les coordonnées avec la commande suivante. 

\commande|set_no_coords|\index{set_no_coords@\lstinline+set_no_coords+}\smallskip

Et la commande inverse aussi existe. 

\commande|set_coords|\index{set_coords@\lstinline+set_coords+}\smallskip


\subsection{Vue blanche ou noire}

\subsection{Noms des joueurs}


\section{Pièces et positions}

\subsection{Réglage du thème des pièces}

\subsection{Lecture de données au format \textsc{pgn}}

\subsection{Lecture de données au format \textsc{fen}}


\section{Annotation}


\commande|test_initbackboard«test»|


\printbibliography
\printindex
\end{document}



%%% Local Variables:
%%% flyspell-mode: 1
%%% ispell-local-dictionary: "american"
%%% End:
